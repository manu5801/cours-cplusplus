\documentclass{beamer}
\usepackage[utf8]{inputenc}
\usepackage{listings}
\usetheme{Warsaw}
\usecolortheme{wolverine}
\usepackage{color}

\definecolor{mygreen}{rgb}{0,0.6,0}
\definecolor{gris}{rgb}{0.9,0.9,0.9}
\definecolor{mymauve}{rgb}{0.58,0,0.82}

\lstset{ %
  backgroundcolor=\color{gris},   % choose the background color; you must add \usepackage{color} or \usepackage{xcolor}
  basicstyle=\footnotesize,        % the size of the fonts that are used for the code
  breakatwhitespace=false,         % sets if automatic breaks should only happen at whitespace
  breaklines=true,                 % sets automatic line breaking
  captionpos=b,                    % sets the caption-position to bottom
  commentstyle=\color{mygreen},    % comment style
  deletekeywords={...},            % if you want to delete keywords from the given language
  escapeinside={\%*}{*)},          % if you want to add LaTeX within your code
  extendedchars=true,              % lets you use non-ASCII characters; for 8-bits encodings only, does not work with UTF-8
  frame=single,	                   % adds a frame around the code
  keepspaces=true,                 % keeps spaces in text, useful for keeping indentation of code (possibly needs columns=flexible)
  keywordstyle=\color{blue},       % keyword style
  language=c++,                 % the language of the code
  otherkeywords={*,...},           % if you want to add more keywords to the set
  %numbers=left,                    % where to put the line-numbers; possible values are (none, left, right)
  %numbersep=5pt,                   % how far the line-numbers are from the code
  %numberstyle=\tiny\color{mygray}, % the style that is used for the line-numbers
  rulecolor=\color{black},         % if not set, the frame-color may be changed on line-breaks within not-black text (e.g. comments (green here))
  showspaces=false,                % show spaces everywhere adding particular underscores; it overrides 'showstringspaces'
  showstringspaces=false,          % underline spaces within strings only
  showtabs=false,                  % show tabs within strings adding particular underscores
  stepnumber=2,                    % the step between two line-numbers. If it's 1, each line will be numbered
  stringstyle=\color{mymauve},     % string literal style
  tabsize=2,	                   % sets default tabsize to 2 spaces
  title=\lstname                   % show the filename of files included with \lstinputlisting; also try caption instead of title
}


\title{Les exceptions}
\subtitle{Introduction au C++ et à la programmation objet}
\author{E. Courcelle}\institute{CALMIP, URA 3667}
\date{Décembre 2019}

\addtobeamertemplate{footline}{\insertframenumber/\inserttotalframenumber}

\begin{document}

\begin{frame}
\titlepage
\end{frame}

\section*{Table des matières}
\begin{frame}
\tableofcontents
\end{frame}

\pgfdeclareimage[height=0.5cm]{logo}{images/cnrs+inpt}
\logo{\pgfuseimage{logo}}

\section{Les exceptions}

\subsection{Le concept}

\begin{frame}[fragile=singleslide,shrink=20]
\frametitle {Que faire en d'erreur ?}

Lorsqu'un objet rencontre une condition d'erreur, que doit-il faire ?
\begin{itemize}
\item{Prendre une décision ? \textbf{Surtout pas !}}
\end{itemize}

L'objet \textbf{ne sait pas} comment l'erreur doit être traitée: tout dépend du contexte, or l'objet doit fonctionner
dans des contextes différents.

L'objet doit:
\begin{itemize}
\item{Soit traiter l'erreur lui-même, s'il le peut}
\item{Soit \textbf{signaler } au programme appelant qu'il a rencontré une situation d'exception}
\end{itemize}
\end{frame}

\begin{frame}[fragile=singleslide,shrink=20]
\frametitle {Comment signaler une condition d'exception ?}

On peut signaler une condition d'exception par trois moyens:
\begin{itemize}
\item{Terminer la fonction en retournant un code d'erreur: \textbf{Pas toujours possible !} (quel code renverra la fonction \texttt{sqrt} ?)}
\item{Déposer un code d'erreur dans une variable globale (\texttt{errno} en C). \textbf{Possible mais lourd}}
\item{Générer une exception; \textbf{Le plus souple}}
\end{itemize}

\end{frame}

\begin{frame}[fragile=singleslide,shrink=20]
\frametitle {Qu'est-ce qu'une exception ?}

C'est un \textbf{objet} qui va contenir des informations sur l'exception à condition que le programmeur les y mette:
\begin{itemize}
\item{Message d'erreur}
\item{Code numérique d'erreur}
\item{Numéro de ligne, nom de fichier, ...}
\end{itemize}

Cet objet sera \textbf{"lancé"} comme une bouteille à la mer par la fonction qui a rencontré la condition d'exception, puis la fonction est interrompue.

L'objet va alors \textbf{remonter la pile d'appels}, et chaque fonction appelante a le choix entre:

\begin{itemize}
\item{Intercepter l'exception et la traiter}
\item{L'ignorer, dans ce cas l'objet passera au niveau supérieur, mais la fonction sera interrompue}
\end{itemize}

Si aucune fonction ne traite l'exception, \textbf{le programme lui-même est interrompu} par le système d'exploitation.

\end{frame}

\subsection{La syntaxe}

\begin{frame}[fragile=singleslide,shrink=20]
\frametitle {Lancer une exception}

La fonction suivante implémente la division d'un complexe par un flottant:
\begin{lstlisting}[language=c++]
complexe& complexe::operator/=(float x) {
  r /= x;
  i /= x;
  return *this;
}
\end{lstlisting}

Si on appelle la fonction ci-dessus avec x=0, le programme s'arrêtera brutalement avec un message d'erreur.
Dans la version ci-dessous, on laisse au programeur la possibilité de gérer ce cas:
\begin{lstlisting}[language=c++]
complexe& complexe::operator/=(float x) {
  if ( x == 0 ) throw ( "division par zero" );  
  r /= x;
  i /= x;
  return *this;
}
\end{lstlisting}
\end{frame}

\begin{frame}[fragile=singleslide,shrink=20]
\frametitle {Attraper et afficher une exception}

La fonction \texttt{main} ci-dessous propose un traitement d'erreur:
\begin{lstlisting}[language=c++]
int main() {
   complexe c(5,6);
   try
   {
   	float x;
   	cout << "Entrez un diviseur: ";
    cin >> x;
    c /= x;
   }
   catch ( const char * e )
   {
    	cout << e << "\n";
   }
   return 0;
}
\end{lstlisting}
\end{frame}

\begin{frame}[fragile=singleslide,shrink=20]
\frametitle {Attraper et traiter une exception}

La fonction \texttt{main} ci-dessous essaie de traiter l'exception et de réagir en conséquence:
\begin{lstlisting}[language=c++]
int main() {
   complexe c(5,6);
   do
   {
      try
      {
         float x;
         cout << "Entrez un diviseur: ";
         cin >> x;
         c /= x;
         break;
      }
      catch ( const char * msg )
      {
         cout << msg << " Recommencez\n";
      }
   } while (true);
   return 0;
}
\end{lstlisting}
\end{frame}

\begin{frame}[fragile=singleslide,shrink=20]
\frametitle {Le traitement peut se faire à différents niveaux}

La fonction \texttt{main} ci-dessous ne traite rien, car le traitement est fait dans la fonction \texttt{inputEtDivise}
\begin{lstlisting}[language=c++]
void inputEtDivise(complexe& c) {
   do
   {
      try
      {
         float x;
         cout << "Entrez un diviseur: ";
         cin >> x;
         c /= x;
         break;
      }
      catch ( const char * msg )
      {
         cout << msg << " Recommencez\n";
      }
   } while (true);
}

int main() {
   complexe c(5,6);
   inputEtDivise(c);
   cout << "Partie reelle    : " << c.get_r() << "\n";
   cout << "Partie imaginaire: " << c.get_i() << "\n";
}
\end{lstlisting}
\end{frame}

\begin{frame}[fragile=singleslide,shrink=20]
\frametitle {La hiérarchie d'exceptions}

\begin{itemize}
\item{Il est préférable d'envoyer des \textbf{objets}, et pas simplement une chaine de caractères.}
\item{Il est par ailleurs \textbf{plus que conseillé} d'utiliser la \textbf{hiérarchie d'exceptions} prédéfinie par la bibliothèque standard du C++}
\end{itemize}

\begin{centering}%
\pgfimage[width=\paperheight]{images/hierarchie-exception}%
\par%
\end{centering}%
\end{frame}

\begin{frame}[fragile=singleslide,shrink=20]
\frametitle {Envoyer un objet exception}

La fonction de division:
\begin{lstlisting}[language=c++]
complexe& complexe::operator/=(float x) {
  if ( x == 0 ) {
     domain_error e ("division par zero" );
     throw e;
  }
  r /= x;
  i /= x;
  return *this;
}
\end{lstlisting}

Soit, en plus concis:
\begin{lstlisting}[language=c++]
complexe& complexe::operator/=(float x) {
  if ( x == 0 ) {
     throw domain_error ("division par zero" );
  }
  r /= x;
  i /= x;
  return *this;
}
\end{lstlisting}
\end{frame}

\begin{frame}[fragile=singleslide,shrink=20]
\frametitle {Des traitements plus ou moins fins}

Une nouvelle version de la première fonction main:
\begin{lstlisting}[language=c++]
int main() {
   complexe c(5,6);
   try
   {
   	float x;
   	cout << "Entrez un diviseur: ";
	cin >> x;
	c /= x;
   }
   catch ( exception & e ) 
   {
    	cout << e.what() << "\n";
   }
   return 0;
}
\end{lstlisting}
\end{frame}

\begin{frame}[fragile=singleslide,shrink=20]
\frametitle {Des traitements plus ou moins fins (2)}

Une nouvelle version de la seconde fonction main:
\begin{lstlisting}[language=c++]
void inputEtDivise(complexe& c) {
	do
	{
      try
      {
         float x;
         cout << "Entrez un diviseur: ";
         cin >> x;
         c /= x;
         break;
      }
      catch ( const domain_error& e )
      {
         cout << e.what() << " Recommencez\n";
      }
   } while (true);
}

int main() {
   complexe c(5,6);
   try
   {
      inputEtDivise(c);
      cout << "Partie reelle    : " << c.get_r() << "\n";
      cout << "Partie imaginaire: " << c.get_i() << "\n";
   }
   catch ( const exception& e )
   {
      cout << e.what() << "\n";
   }
}
\end{lstlisting}
\end{frame}

\begin{frame}[fragile=singleslide,shrink=20]
\frametitle {Des traitements plus ou moins fins}

Dans les codes qui précèdent, l'utilisation du caractère \texttt{\&} est \textbf{fondamentale}: en effet, passer l'exception par 
référence permet:

\begin{itemize}
\item{D'utiliser l'édition de lien dynamique et non pas statique}
\item{D'ajuster les traitements à l'exception \textbf{réellement émise}}
\item{D'avoir des traitements plus ou moins fins, suivant qu'on attrape l'exception plus ou moins haut dans la 
chaine d'héritage}
\end{itemize}

\textbf{Il est indispensable que toutes les exceptions émises dérivent d'un même objet}

Les exceptions non capturées ne sont \textbf{pas traitées} et remontent toute la chaine d'appels, 
éventuellement jusqu'au système d'exploitation (arrêt du programme).

\end{frame}

\begin{frame}[fragile=singleslide,shrink=20]
\frametitle {traitement de plusieurs types d'exceptions}

Le programme suivant traite toutes les exceptions possibles:
\begin{lstlisting}[language=c++]
try {
    blabla
  }
  catch (const  & domain_error e) {
    blabla
  }
  catch (const  & bad_alloc e) {
     blabla
  }  
  catch (...) {
    cout << "Autre exception\n";
  };
} 
\end{lstlisting}
\end{frame}

\begin{frame}[fragile=singleslide,shrink=20]
\frametitle {Exceptions et constructeurs}

\begin{centering}
\textbf{OUIIIIIIIIIIIIIIIIIIIIIIIIIIII !}
\end{centering}

\end{frame}

\begin{frame}[fragile=singleslide,shrink=20]
\frametitle {Exceptions et destructeurs}

\begin{centering}
\textbf{NOOOOOOOOOOOOOOOOOOOOOOOOOOOOON !}
\end{centering}

\end{frame}

\begin{frame}
(à suivre)
\end{frame}

\end{document}
