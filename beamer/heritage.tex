\documentclass{beamer}
\usepackage[utf8]{inputenc}
\usepackage{listings}
\usetheme{Warsaw}
\usecolortheme{wolverine}
\usepackage{color}

\definecolor{mygreen}{rgb}{0,0.6,0}
\definecolor{gris}{rgb}{0.9,0.9,0.9}
\definecolor{mymauve}{rgb}{0.58,0,0.82}

\lstset{ %
  backgroundcolor=\color{gris},   % choose the background color; you must add \usepackage{color} or \usepackage{xcolor}
  basicstyle=\footnotesize,        % the size of the fonts that are used for the code
  breakatwhitespace=false,         % sets if automatic breaks should only happen at whitespace
  breaklines=true,                 % sets automatic line breaking
  captionpos=b,                    % sets the caption-position to bottom
  commentstyle=\color{mygreen},    % comment style
  deletekeywords={...},            % if you want to delete keywords from the given language
  escapeinside={\%*}{*)},          % if you want to add LaTeX within your code
  extendedchars=true,              % lets you use non-ASCII characters; for 8-bits encodings only, does not work with UTF-8
  frame=single,	                   % adds a frame around the code
  keepspaces=true,                 % keeps spaces in text, useful for keeping indentation of code (possibly needs columns=flexible)
  keywordstyle=\color{blue},       % keyword style
  language=c++,                 % the language of the code
  otherkeywords={*,...},           % if you want to add more keywords to the set
  %numbers=left,                    % where to put the line-numbers; possible values are (none, left, right)
  %numbersep=5pt,                   % how far the line-numbers are from the code
  %numberstyle=\tiny\color{mygray}, % the style that is used for the line-numbers
  rulecolor=\color{black},         % if not set, the frame-color may be changed on line-breaks within not-black text (e.g. comments (green here))
  showspaces=false,                % show spaces everywhere adding particular underscores; it overrides 'showstringspaces'
  showstringspaces=false,          % underline spaces within strings only
  showtabs=false,                  % show tabs within strings adding particular underscores
  stepnumber=2,                    % the step between two line-numbers. If it's 1, each line will be numbered
  stringstyle=\color{mymauve},     % string literal style
  tabsize=2,	                   % sets default tabsize to 2 spaces
  title=\lstname                   % show the filename of files included with \lstinputlisting; also try caption instead of title
}


\title{L'héritage}
\subtitle{Introduction au C++ et à la programmation objet}
\author{E. Courcelle}\institute{CALMIP, URA 3667}
\date{Mai 2022}

\pgfdeclareimage[height=0.5cm]{logo}{images/cnrs+inpt}
\logo{\pgfuseimage{logo}}

\addtobeamertemplate{footline}{\insertframenumber/\inserttotalframenumber}

\begin{document}

\begin{frame}
\titlepage
\end{frame}

\begin{frame}
\tableofcontents
\end{frame}

\section{Classes abstraites et concrètes}

\begin{frame}[fragile=singleslide,shrink=20]
\frametitle {Une classe abstraite}

\begin{block}{Une machine à café, ça... fait le café}
...mais on ne précise pas comment. \em{Version diagramme}
\end{block}

\begin{centering}%
\pgfimage[width=\paperheight]{images/cafetiere}%
\par%
\end{centering}%
\end{frame}

\begin{frame}[fragile=singleslide,shrink=20]
\frametitle {Une classe abstraite}

\begin{block}{Une machine à café, ça... fait le café}
...mais on ne précise pas comment. \em{Version code}
\end{block}
\begin{lstlisting}[language=c++]
class Cafetiere {
   public:
      virtual void faire_le_cafe() = 0;
      virtual int lire_etat() = 0;
      void encaisser_monnaie();

   protected:
      void verser_eau();
      void sucre();
      void donner_gobelet();
      void donner_cuiller();
      
   private:
      ReservoirEau eau;
      ReservoirSucre sucre;
      ReservoirGobelets gobelets;
      ReservoirCuillers cuillers;
}
\end{lstlisting}
\end{frame}

\begin{frame}[fragile=singleslide,shrink=20]
\frametitle {Plusieurs classes concrètes}

\begin{block}{Plusieurs manière de faire le café}
...Les classes concrètes précisent les choses. \em{Version diagramme}
\end{block}

\begin{centering}%
\pgfimage[width=\paperheight]{images/cafetiere2}%
\par%
\end{centering}%
\end{frame}

\begin{frame}[fragile=singleslide,shrink=20]
\frametitle {Plusieurs classes concrètes}

\begin{block}{Plusieurs manière de faire le café}
...Les classes concrètes précisent les choses. \em{Version code}
\end{block}

\begin{lstlisting}[language=c++]
class CafetiereSoluble: public Cafetiere {

public:
  void faire_le_cafe() override;
  int lire_etat() override;

protected:
  void verser_cafe();
  
private:
  ReservoirCafeSoluble cafe;
}

class CafetierePoudre: public Cafetiere {

public:
  void faire_le_cafe() override;
  int lire_etat() override;

protected:
  void verser_cafe_dans_filtre();
  
private:
  ReservoirCafePoudre cafe;
}
\end{lstlisting}

\end{frame}

\begin{frame}[fragile=singleslide,shrink=20]
\frametitle {Sections private, protected, public}

\begin{block}{private}
\begin{itemize}
\item{Données encapsulées par l'objet}
\item{Fonctions membres correspondant au fonctionnement interne de l'objet.}
\end{itemize}
\em{On peut les modifier comme on veut !}
\end{block}

\begin{block}{protected}
\begin{itemize}
\item{Pas de données}
\item{Fonctions membres accessibles par les objets dérivés, \textbf{et seulement eux}.}
\end{itemize}
\em{Attention aux modifications, risque de casser le code des classes dérivées} (mais ça reste relativement localisé)
\end{block}

\begin{block}{public}
\begin{itemize}
\item{Pas de données}
\item{Fonctions membres accessibles par le code utilisateur de l'objet}.
\end{itemize}
\em{ ON NE CHANGE RIEN, risque de casser plein de code utilisateur}
\end{block}

\end{frame}

\begin{frame}[fragile=singleslide,shrink=20]
\frametitle {Constructeurs par défaut}

\begin{lstlisting}[language=c++]
CafetiereSoluble ma_cafetiere;
\end{lstlisting}

\begin{block}{Que se passe-t-il ?}
\begin{enumerate}
\item Allocation de mémoire pour \tt{Cafetiere} \bf{et} \tt{CafetiereSoluble}
\item Appel du \bf{constructeur par défaut} de \tt{Cafetiere}
\item Appel du \bf{constructeur par défaut} de \tt{CafetiereSoluble}
\end{enumerate}
\end{block}
\end{frame}

\begin{frame}[fragile=singleslide,shrink=20]
\frametitle {Passage de paramètres aux constructeurs}

\begin{block}{Un constructeur peut en appeler un autre}
Le constructeur de la classe dérivée \bf{appelle explicitement} le constructeur de la classe de base.

\begin{enumerate}
\item Allocation de mémoire pour \tt{Cafetiere} \bf{et} \tt{CafetiereSoluble}
\item Appel du \bf{constructeur} de \tt{Cafetiere} \bf{avec ses paramètres}
\item Initialisation des membres de \tt{CafetiereSoluble}
\item Appel du code du constructeur de \tt{CafetiereSoluble}
\end{enumerate}
\end{block}

\begin{lstlisting}[language=c++]
class Cafetiere {
public:
   Cafetiere(float e, int g, int c,float s) : eau(e), gobelets(g), cuillers(c), sucre(s) {};
...
}
\end{lstlisting}

\begin{lstlisting}[language=c++]
class CafetiereSoluble: public Cafetiere {
public:
   CafetiereSoluble(float e, int g, int c,float s, float f) : Cafetiere(e,g,c,s), cafe(f) {};
...
}
}
\end{lstlisting}
\end{frame}

\begin{frame}[fragile=singleslide,shrink=20]
\frametitle {Le destructeur}

\begin{block}{C'est pareil, sauf que c'est l'inverse}

\begin{enumerate}
\item Appel du code du destructeur de \tt{CafetiereSoluble}
\item Appel du code du destructeur de \tt{Cafetiere}
\item Désallocation de la mémoire
\end{enumerate}
\end{block}

\begin{block}{...mais c'est comme les antibiotiques: c'est pas automatique !}
Il faut définir pour chaque classe de la hiérarchie
un destructeur virtuel. Il peut ne rien faire (\{\}), mais il garantit que le destructeur de la classe de base est bien appelé.
\end{block}

\begin{lstlisting}[language=c++]
class Cafetiere {
public:
   virtual ~Cafetiere(){};
   ...
}
classe CafetiereSoluble: public Cafetiere {
    ~CafetiereSoluble(){};
}
\end{lstlisting}
\end{frame}

\section{Le polymorphisme}

\begin{frame}[fragile=singleslide,shrink=20]
\frametitle {Déclarations}

\begin{block}{On peut pas !}
\begin{lstlisting}[language=c++]
Cafetiere C;
\end{lstlisting}
\end{block}

\begin{block}{On peut !}
\begin{lstlisting}[language=c++]
Cafetiere* C;
\end{lstlisting}
\end{block}

\begin{block}{On peut !}
\begin{lstlisting}[language=c++]
void ma_fonction(Cafetiere& c);

CafetiereSoluble cs;
CafetierePoudre cp;
ma_fonction(cs);
ma_fonction(cp);
\end{lstlisting}
\end{block}
\end{frame}

\begin{frame}[fragile=singleslide,shrink=20]
\frametitle {Un vecteur polymorphe}

\begin{block}{Une cafeteria}
Le code ci-dessous permet de déclarer un vecteur (tableau) polymorphe, c'est-à-dire un vecteur contenant des objets de plusieurs types.
La notation est un peu lourde, elle fait appel à la stl, \bf{mais c'est la bonne manière de faire}.
\end{block}

\begin{lstlisting}[language=c++]
vector<shared_ptr<Cafetiere>> machines;
machines.push_back(make_shared<CafetiereSoluble>());
machines.push_back(make_shared<CafetiereSoluble>());
machines.push_back(make_shared<CafetierePoudre>());

for (auto m : machines) {
    if (m->lire_etat() == true) {
       m->faire_le_cafe();
       m->encaisser_monnaie();
    };
};
\end{lstlisting}
\end{frame}

\begin{frame}[fragile=singleslide,shrink=20]
\frametitle {Le clonage polymorphique}

\begin{block}{Sémantique d'entité}
Les objets dont nous parlons ont une sémantique d'entité, ils n'ont pas d'operator=, par contre ils
peuvent implémenter le clonage polymorphique.
\end{block}

\begin{lstlisting}[language=c++]
class Cafetiere {
   ...
   virtual Cafetiere * Cafetiere clone() = 0;
}

class CafetiereSoluble {
   ...
   CafetiereSoluble * Cafetiere clone() override
   {
       return new CafetiereSoluble(*this);
   }
   
};
\end{lstlisting}
\end{frame}

\end{document}
  
