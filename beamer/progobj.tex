\documentclass{beamer}
\usepackage[utf8]{inputenc}
\usepackage{listings}
\usetheme{Warsaw}
\usecolortheme{wolverine}
\usepackage{color}

\definecolor{mygreen}{rgb}{0,0.6,0}
\definecolor{gris}{rgb}{0.9,0.9,0.9}
\definecolor{mymauve}{rgb}{0.58,0,0.82}

\lstset{ %
  backgroundcolor=\color{gris},   % choose the background color; you must add \usepackage{color} or \usepackage{xcolor}
  basicstyle=\footnotesize,        % the size of the fonts that are used for the code
  breakatwhitespace=false,         % sets if automatic breaks should only happen at whitespace
  breaklines=true,                 % sets automatic line breaking
  captionpos=b,                    % sets the caption-position to bottom
  commentstyle=\color{mygreen},    % comment style
  deletekeywords={...},            % if you want to delete keywords from the given language
  escapeinside={\%*}{*)},          % if you want to add LaTeX within your code
  extendedchars=true,              % lets you use non-ASCII characters; for 8-bits encodings only, does not work with UTF-8
  frame=single,	                   % adds a frame around the code
  keepspaces=true,                 % keeps spaces in text, useful for keeping indentation of code (possibly needs columns=flexible)
  keywordstyle=\color{blue},       % keyword style
  language=c++,                 % the language of the code
  otherkeywords={*,...},           % if you want to add more keywords to the set
  %numbers=left,                    % where to put the line-numbers; possible values are (none, left, right)
  %numbersep=5pt,                   % how far the line-numbers are from the code
  %numberstyle=\tiny\color{mygray}, % the style that is used for the line-numbers
  rulecolor=\color{black},         % if not set, the frame-color may be changed on line-breaks within not-black text (e.g. comments (green here))
  showspaces=false,                % show spaces everywhere adding particular underscores; it overrides 'showstringspaces'
  showstringspaces=false,          % underline spaces within strings only
  showtabs=false,                  % show tabs within strings adding particular underscores
  stepnumber=2,                    % the step between two line-numbers. If it's 1, each line will be numbered
  stringstyle=\color{mymauve},     % string literal style
  tabsize=2,	                   % sets default tabsize to 2 spaces
  title=\lstname                   % show the filename of files included with \lstinputlisting; also try caption instead of title
}


\title{La programmation orientée objets}
\subtitle{Introduction au C++ et à la programmation objet}
\author{E. Courcelle}\institute{CALMIP, UMS 3669}
\date{Mai 2016}

\pgfdeclareimage[height=0.5cm]{logo}{images/cnrs+inpt}
\logo{\pgfuseimage{logo}}

\addtobeamertemplate{footline}{\insertframenumber/\inserttotalframenumber}

\begin{document}

\begin{frame}
\titlepage
\end{frame}

\section*{Table des matières}
\begin{frame}
\tableofcontents
\end{frame}

\section{Les modules, ancêtres des objets}

\subsection{Variables locales, variables globales}

\begin{frame}[fragile=singleslide,shrink=20]
\frametitle {Une variable globale...}
Une \textbf{variable globale} est accessible à partir de \\ \textbf{toutes les fonctions du programme}
\begin{lstlisting}[language=c++]
  %*\textbf{int x=0};*)
int fonction1() {
  %*\textbf{x=1;}*)
  return 2 * x;         
}
int fonction2() {
  %*\textbf{x = x + 2;}*)
  return 2 * x;         
}  
\end{lstlisting}
\end{frame}

\begin{frame}[fragile=singleslide,shrink=20]
\frametitle {...Une variable locale}
Une \textbf{variable locale} n'est accessible qu'à partir d'un \textbf{bloc déterminé} \\
(par exemple une \texttt{fonction}):
\begin{lstlisting}[language=c++]
int fonction() {
  %*\textbf{int x=1;}*)
  return 2 * x;         
}
\end{lstlisting}
\end{frame}

\subsection{L'approche modulaire}

\begin{frame}
\frametitle {Un module: Interface et implémentation}

\begin{columns}[t]

\begin{column}{6cm}
\textbf{Interface du module}
\begin{itemize}
  \item{Noms des fonctions}
  \item{Ce qu'elles font}
  \item{Ce qui rentre \\ (les paramètres des fonctions)}
  \item{Ce qu'elles renvoient \\ (le type de la valeur de retour)}
\end{itemize}
\end{column}

\begin{column}{6cm}
\textbf{Implémentation du module}
\begin{itemize}
  \item{Variables du module \\ ("globales")}
  \item{Code des fonctions}
\end{itemize}
\end{column}

\end{columns}

\end{frame}
  
\begin{frame}
\frametitle {Prototypage et travail en équipe}
\textbf{L'encapsulation} des données permet:
\begin{itemize}
  \item De travailler \alert{par prototypage} en commençant \\ par l'interface publique
  \item De travailler en équipe:
  \begin{itemize}
    \item On se met d'accord \alert{sur l'interface}
    \item Chacun écrit l'implémentation \alert{"comme il/elle veut"}
  \end{itemize}
\end{itemize}
\end{frame}

\section{Les objets}
\subsection{Une pile de char}
  
\begin{frame}[fragile=singleslide,shrink=20]
\frametitle {Un module qui définit une pile de caractères}
L'interface du module est constitué par deux fonctions: l'une pour empiler, l'autre pour dépiler.
\begin{lstlisting}[language=c++]
// Mettre un caractere en haut de la pile 
// et la faire grossir d'un caractere
void push_char(char);
  
// Renvoyer le caractere du haut de la pile 
// et diminuer la pile d'un caractere
char pop_char();  
\end{lstlisting} 
\end{frame}
  
\begin{frame}[fragile=singleslide,shrink=20]
\frametitle {Gérer plusieurs piles de caractères}
On ajoute un entier qui permettra d'identifier la pile:
\begin{lstlisting}[language=c++]
// Mettre un caractere en haut de la pile 
// et la faire grossir d'un caractere
// id permet de choisir une pile parmi plusieurs
void push_char(int id, char c);
  
// Renvoyer le caractere du haut de la pile 
// et diminuer la pile d'un caractere
// id permet de choisir une pile parmi plusieurs
char pop_char(int id);  
\end{lstlisting} 
\end{frame}

\subsection{Des objets intégrés au système de typage}

\begin{frame}[fragile=singleslide,shrink=20]
\frametitle {Créer un nouveau type de variables}
La solution du C++: créer un \alert{nouveau type} de variables

\begin{lstlisting}[language=c++]
class Stack_char {

// Mettre un caractere en haut de la pile 
// et la faire grossir d'un caractere
void push(char c);

// Renvoyer le caractere du haut de la pile 
// et diminuer la pile d'un caractere
char pop();

}

// Creer trois piles dans notre programme
Stack_char a,b,c;

a.push('a');
a.push('z');
b.push('u');
c.push('t');
cout << a.pop(); // Imprime %*\textbf{z}*)
\end{lstlisting} 
\end{frame}
  
\subsection {Opérations sur les objets}

\begin{frame}[fragile=singleslide,shrink=20]
\frametitle{Comme un type natif !}
Un tableau de piles de caractères:
\begin{lstlisting}[language=c++]
Stack_char[10] tableau_de_piles;
\end{lstlisting}
Une pile initialisée (c++11):
\begin{lstlisting}[language=c++]
Stack_char a = {'a','z','e','r','t','y'};
\end{lstlisting}
Recopier une pile dans une autre:
\begin{lstlisting}[language=c++]
Stack_char b = a;
\end{lstlisting}
Effectuer un transtypage:
\begin{lstlisting}[language=c++]
Stack_char a;
Stack_int b = (Stack_int) a;
\end{lstlisting}
\textbf{... à condition de définir ce que tout ça signifie !}
\end{frame}

\section{L'héritage}
\subsection {Classer les objets}

\begin{frame}
\frametitle{Ciel, quel désordre dans mon argenterie !}    
\begin{centering}%
\pgfimage[width=\paperheight]{images/bazar}%
\par%
\end{centering}%
\end{frame}

\begin{frame}
\frametitle{C'est mieux comme ça !}
\begin{centering}
\pgfimage[width=\paperheight]{images/range}%
\par%
\end{centering}%
\end{frame}

\begin{frame}
\frametitle{C'est mieux comme ça !}
\begin{centering}
\pgfimage[width=\paperheight]{images/range-uml-0}%
\par%
\end{centering}%
\end{frame}

\begin{frame}
\frametitle{C'est mieux comme ça !}
\begin{centering}
\pgfimage[width=\paperheight]{images/range-uml-1}%
\par%
\end{centering}%
\end{frame}

\subsection{Dessiner des formes}
\begin{frame}[fragile=singleslide,shrink=20]
\frametitle{Plusieurs formes}
On voudrait faire un programe pour dessiner plusieurs formes: \\
Créons donc plusieurs types d'objets:
\begin{lstlisting}[language=c++]
class Circle;
class Triangle;
class Square:
\end{lstlisting}
Dessin de formes... et d'une pile ! \\
mouais... Mais que ce soient des formes ou des piles, on mettrait tout sur le même plan ?
\end{frame}

\begin{frame}[fragile=singleslide,shrink=20]
\frametitle{NON ! Une forme est une forme !}
Une forme est un objet qu'on peut \textbf{dessiner} et qui a \textbf{un centre} et \textbf{une couleur}.\\
Et il y en a \textbf{plusieurs sortes}.
\begin{lstlisting}[language=c++]
enum kind {circle, triangle, square};
class Shape {
   point center;
   color col;
   kind k;
public:
   point where() {return center};
   void draw();
};
\end{lstlisting}
Voilà qui est déjà mieux ! 
\end{frame}

\begin{frame}[fragile=singleslide,shrink=20]
\frametitle{La fonction draw}
Imaginons à quoi va ressembler la fonction draw:
\begin{lstlisting}[language=c++]
void Shape::draw() {
  if (kind==cercle) {
     // dessine-moi un cercle !
  } else if (kind==triangle) {
     // dessine-moi un triangle !
  } else if (kind==square) {
     // dessine-moi un carre !
  } else {
     // oups je ne sais pas dessiner cela !
  }
\end{lstlisting}
Pour le passage à l'échelle on repassera ! \\
Plus il y a de formes à dessiner, plus la fonction draw deviendra compliquée et illisible !
\end{frame}
  
\begin{frame}[fragile=singleslide,shrink=20]
\frametitle{Ben OUI ! Une forme est... une forme !}
Reprenons: une forme, c'est un truc qu'on peut dessiner, \\ qui a une couleur et un centre. \\
Sauf que je ne sais pas à quoi elle ressemble, donc je ne sais pas le dessiner. \\
En c++, ça s'écrit comme ça:

\begin{lstlisting}[language=c++]
class Shape {
private:
   point center;
   color col;
public:
   point where() { return center;};
   %*\textbf{virtual}*) void draw()=0;
}
\end{lstlisting}
draw est une \textbf{fonction virtuelle pure}, on sait qu'elle existera un jour, mais on ne sait pas ce qu'elle fera !
\end{frame}

\begin{frame}[fragile=singleslide,shrink=20]
\frametitle{Une classe abstraite...}
Les objets de classe Shape ne \textbf{ne peuvent pas} être créés ! \\
C'est normal puisque Shape est une classe virtuelle: 
\begin{lstlisting}[language=c++]
Shape une_forme; // NE COMPILE PAS !
\end{lstlisting}

Par contre, je peux imaginer une fonction qui prenne une forme en paramètre:
\begin{lstlisting}[language=c++]
void arriere_plan(const & Shape); // PAS DE PROBLEME !
\end{lstlisting}
Je ne peux pas créer une forme, mais si j'en ai une je peux la passer à une fonction... \\
Mais comment pourrais-je avoir une forme ?
\end{frame}

\begin{frame}[fragile=singleslide,shrink=20]
\frametitle{...Des classes concrètes...}
Mais à quoi sert donc ce concept de forme, si ce n'est à dessiner \\ 
des cercles, des triangles, des carrés ? \\
On va donc écrire en c++ que carre, cercle, triangle sont \textbf{des sortes de} forme:

\begin{lstlisting}[language=c++]
class Circle: public Shape {
   ...
public:
   virtual void draw();
};
class Triangle: public Shape {
   ...
public:
   virtual void draw();
};
class Square: public Shape {
   ...
public:
   virtual void draw();
};
\end{lstlisting}
\end{frame}

\begin{frame}[fragile=singleslide,shrink=20]
\frametitle{...qu'on sait dessiner et créer !}

\begin{block}{Pour utiliser une forme on fait comme d'habitude:}
\begin{itemize}
\item{Déclarer les variables en leur donnant le type adéquat}
\item{Appeler la fonction \texttt{draw} pour dessiner}
\end{itemize}
\begin{lstlisting}[language=c++]
Circle c;
Triangle d;
Square s;
c.draw(); d.draw(); s.draw()
\end{lstlisting}
\end{block}

\begin{block}{Pour créer une nouvelle forme on doit:}
\begin{itemize}
\item{Créer une classe qui dérive de Shape}
\item{Implémenter la fonction draw associée}
\item{Ainsi on ne touche pas à ce qui existe déjà !}
\item{...Et maintenant ça passe à l'échelle}
\end{itemize}
\end{block}
\end{frame}
  
\begin{frame}
\frametitle{Héritage correct}
L'héritage permettra de faire évoluer votre code en douceur... \\ 
\textbf{à condition qu'il soit correctement défini !}
  
\begin{block}{Héritage correct}
Les fonctions virtuelles des classes dérivées doivent:
\begin{itemize}
\item{En faire au moins autant que la fonction de la classe de base}
\item{Ne pas avoir d'exigence supérieure à celles de la classe de base}
\end{itemize}
\end{block}
\end{frame}

\section{Résumé}
\subsection{Donner un sens à ses classes}

\begin{frame}
\frametitle{Donner un sens à ses classes}
\begin{block}{Des classes chargées de sens}
Trois types principaux, du point-de-vue de la sémantique
\begin{itemize}
\item{Sémantique de \textbf{valeur}}
\item{Sémantique d'\textbf{entité}}
\item{Gestionnaires de ressources}
\end{itemize}
\end{block}
\end{frame}


\section{Résumé}
\subsection{Classes et objets}

\begin{frame}
\frametitle{Classes et objets}
\begin{block}{Qu'est-ce qu'une classe ?}
Une classe est la description d'un type d'objets.
\begin{itemize}
\item{C'est en quelque sorte un \textbf{moule}}
\item{Il permettra par la suite de créer de nouveaux objets}
\item{Techniquement, c'est un type de données}
\end{itemize}
\end{block}
\begin{block}{Qu'est-ce qu'un objet ?}
Un objet est une \textbf{variable} du programme, qui se caractérise par :
\begin{itemize}
\item{Un état}
\item{Un comportement}
\item{Une identité}
\end{itemize}
\end{block}
\end{frame}

\end{document}
  
