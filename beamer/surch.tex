\documentclass{beamer}
\usepackage[utf8]{inputenc}
\usepackage{listings}
\usetheme{Warsaw}
\usecolortheme{wolverine}
\usepackage{color}

\input{lstset.tex}

\title{Surcharge des fonctions et opérateurs}
\subtitle{Introduction au C++ et à la programmation objet}
\author{E. Courcelle}\institute{CALMIP, UMS 3669}
\date{Décembre 2019}

\addtobeamertemplate{footline}{\insertframenumber/\inserttotalframenumber}

\begin{document}

\begin{frame}
\titlepage
\end{frame}

\section*{Table des matières}
\begin{frame}
\tableofcontents
\end{frame}

\pgfdeclareimage[height=0.5cm]{logo}{images/cnrs+inpt}
\logo{\pgfuseimage{logo}}

\section{Surcharger les fonctions}

\begin{frame}[fragile=singleslide,shrink=20]
\frametitle {Surcharger une fonction ou une méthode}

En C++, il est possible de déclarer et définir \textbf{plusieurs fonctions ayant le même nom}, à condition que les listes de leurs arguments diffèrent (types différents):

\begin{lstlisting}[language=c++]
float flineaire (float x) {                   
   float y = 3 * x + 2;
   return y;
}

complexe flineaire (const complexe & x) {
   complexe y(0,0);
   y.set_r (3*x.get_r() + 2);
   y.set_i (3*x.get_i());
   return y;
}
\end{lstlisting}

\begin{block}{Les règles de surcharge}
\begin{itemize}
\item{Le type de la valeur de retour ne fait pas partie du protoype: Ainsi, il n'est pas possible de surcharger une fonction par une autre fonction qui aurait même nom et même liste d'arguments mais une valeur de retour différente.}
\item{Dans le cas de méthodes d'objets, \texttt{const} fait partie du prototype, de sorte qu'on peut avoir une méthode \texttt{float f()} et une méthode \texttt{float f() const} }
\item{La norme définit les règles permettant à partir d'une liste de paramètres de choisir l'une ou l'autre des fonctions.}
\end{itemize}
\end{block}
\end{frame}

\begin{frame}[fragile=singleslide,shrink=20]
\frametitle {Surcharger le constructeur}

Le constructeur étant une fonction-membre parmi d'autres, il est possible de le surcharger. 

Dans le cas du type complexe, cela permettra
de remplacer la fonction-membre copie par le \textbf{constructeur de copie}.

Dès lors il est possible d'écrire \texttt{complexe a = b;}

\begin{lstlisting}[language=c++]
class complexe {
public:
   complexe(float x,float y) : r(x), i(y) {};
   complexe(const complexe& c) : r(c.r),i(c.i) {};
private:
   float r;
   float i;
   ...
}

main() {
   const complexe j(0,1);
   complexe A=j;
}
\end{lstlisting}
\end{frame}

\section{Valeurs par défaut des arguments}

\begin{frame}[fragile=singleslide,shrink=20]
\frametitle {Cas des fonctions ou des méthodes}
On peut donner aux arguments des valeurs par défaut.

Une utilisation possible de cela est de réutiliser du code ancien, en étendant une fonction sans remettre en cause l'existant.

Par exemple avec la fonction \texttt{flineaire}:

\begin{lstlisting}[language=c++]
float flineaire (float x, int a=3, int b=2) {                   
   float y = a * x + b;
   return y;
}

float c1 = flineaire(2.0);      // c1 vaut 8.0
float c2 = flineaire(2.0,4);    // c2 vaut 10.0
float c3 = flineaire(2.0,4,4);  // c3 vaut 12.0
\end{lstlisting}

\begin{block}{ATTENTION}
\begin{itemize}
\item{Il n'y a pas d'arguments nommés en C++}
\item{En conséquence, les arguments ayant une valeur par défaut sont \textbf{systématiquement} en fin de liste}
\end{itemize}
\end{block}
\end{frame}

\begin{frame}[fragile=singleslide,shrink=20]
\frametitle {Cas des constructeurs}
On peut donner aux arguments des constructeurs des valeurs par défaut:
\begin{lstlisting}[language=c++]
class complexe {
private:
  ...
public:
  complexe(float x=0, float y=0) {r=x;   i=y; _calc_module();};
  ...
};

main() {
  complexe c;        // sous-entendu initialiser a 0
  complexe c1(2);    // sous-entendu initialiser a (2,0) [reel]
  complexe c2 = {2}; // meme chose que ci-dessus
  complexe c3 = {2,3};
}
\end{lstlisting}
\end{frame}

\begin{frame}[fragile=singleslide,shrink=20]
\frametitle {Conversions de types}

On peut écrire et c'est une \textbf{facilité d'écriture}:
\begin{lstlisting}[language=c++]
   complexe C1 = 2;
\end{lstlisting}

Ce qui correspond à une \textbf{conversion} entre types (ici conversion depuis le type \texttt{float} vers le type \texttt{complexe}). 
Cette conversion peut aussi bien être nuisible si elle n'a pas de sens, 
elle peut dans ce cas être inhibée grâce au mot-clé \texttt{explicit}. 

Cela sera utile par exemple dans la définition de l'objet \texttt{tableau}:

\begin{lstlisting}[language=c++]
class tableau {
  ...
public:
  explicit tableau(int);                                              
};

main() {
  tableau T = 1000;   // Ne compile pas !
  tableau T(1000);    // OK, cree un tableau de dimension 1000
}
\end{lstlisting}

\begin{block}{ATTENTION, Piège !}
Si on ne met pas \texttt{explicit} dans le code ci-dessus, on risque d'avoir des conversions non souhaitées, qui conduisent à des erreurs à l'exécution. 
Ces erreurs seront détectées à la compilation si on les interdit par l'utilisation de \texttt{explicit}
\end{block}
\end{frame}

\section{Surcharger les opérateurs}
\begin{frame}[fragile=singleslide,shrink=20]
\frametitle {Opérateurs et fonctions}
Un opérateur est un appel de fonction écrit de manière plus lisible:

\begin{lstlisting}[language=c++]
c = a + b:
\end{lstlisting}

pourrait s'écrire:

\begin{lstlisting}[language=c++]
c = add(a,b);
\end{lstlisting}
 mais ce ne serait pas très agréable à lire, surtout dans le cas de formules compliquées:
\begin{lstlisting}[language=c++]
d = a + 2*(b+c);            // Ecrit sous forme de formule
d = add(a,mul(2,add(b,c))); // Moins agreable a lire
\end{lstlisting}
\end{frame}

\begin{frame}[fragile=singleslide,shrink=20]
\frametitle {Opérateurs et fonctions (2)}

\begin{itemize}
\item{Surcharger un opérateur revient à surcharger une fonction.}
\item{Cela permettra d'écrire des formules en utilisant des objets}
\item{Par exemple \texttt{c = a + b} où a,b,c sont des complexes}
\end{itemize}

\begin{block}{ATTENTION !!! Pièges !}
\begin{itemize}
\item{La surcharge des opérateurs ne \textbf{change pas les règles de priorité ou d'associativité}}
\item{A chaque opérateur sa fonction avec son \textbf{nom réservé}}
\item{Vous pouvez surcharger les opérateurs existants, \textbf{pas} en inventer de nouveaux !}
\item{On peut mettre n'importe quoi dans le code de la fonction surchargée.

C'est au programeur de \textbf{surcharger ses opérateurs intelligemment !}}
\end{itemize}
\end{block}
\end{frame} 

\begin{frame}[fragile=singleslide,shrink=20]
\frametitle {Les quatre opérations}

\begin{block}{Opérateurs unaires}
\begin{lstlisting}[language=c++]
+=  operator+=
-=  operator-=
*=  operator*=
/=  operator/=
%=  operator%=
\end{lstlisting}
\end{block}

\begin{block}{Opérateurs binaires}
\begin{lstlisting}[language=c++]
+  operator+
-  operator-
*  operator*
/  operator/
%  operator%
\end{lstlisting}
\end{block}
\end{frame}

\begin{frame}[fragile=singleslide,shrink=20]
\frametitle {L'addition de complexes}

Voici une première manière de définir les opérateurs \texttt{+} et \texttt{+=} sur les complexes.

La fonction operator+ fait appel à une fonction amie, pour lui permettre d'accéder aux membres privés.
\begin{lstlisting}[language=c++]
class complexe {
private:
  ...
public:
  ...
  complexe& operator+= (const complexe&);
  friend complexe operator+(const complexe& a, const complexe& b);
};

complexe& complexe::operator+=(const complexe& c) {
  r += c.r;
  i += c.i;
  return *this;
};
complexe operator+(const complexe& a, const complexe& b) {
  complexe c;
  c.r = a.r + b.r;
  c.i = a.i + b.i;
  return c;
}
\end{lstlisting}
\end{frame}

\begin{frame}[fragile=singleslide,shrink=20]
\frametitle {L'addition de complexes (2)}

Voici une seconde manière de définir les opérateurs \texttt{+} et \texttt{+=} sur les complexes.

Elle est préférable à la précédente, car il y a moins de dépendences (et la fonction n'est plus une amie):
\begin{lstlisting}[language=c++]
class complexe {
private:
  ...
public:
  ...
  complexe& operator+= (const complexe&);
};

complexe& complexe::operator+=(const complexe& c) {
  r += c.r;
  i += c.i;
  return *this;
};
complexe operator+(const complexe& a, const complexe& b) {
  complexe c=a;
  c += b;
  return c;
}
\end{lstlisting}
\end{frame}

\begin{frame}[fragile=singleslide,shrink=20]
\frametitle {Incrémentation, décrémentation}

Ces opérateurs (- - ) et (+ +) sont utiles pour définir des \textbf{itérateurs}, permettant d'itérer sur des objets de type conteneur.

Il y a cependant une subtilité importante, utile pour reproduire exactement le comportement des opérateurs natifs du C correspondants:

\textbf{Postindexation:}
\begin{lstlisting}[language=c++]
int i = 1;
j = i++;          // Incremente et renvoie le vieux i
cout << i << j;   // Renvoie 2 1
\end{lstlisting}

\textbf{Préindexation:}
\begin{lstlisting}[language=c++]
int i = 1;
j = ++i;          // Incremente et renvoie le nouveau i
cout << i << j;   // Renvoie 2 2
\end{lstlisting}
\end{frame}

\begin{frame}[fragile=singleslide,shrink=20]
\frametitle {Incrémentation, décrémentation (2)}

En C++, un opérateur d'incrémentation sur un objet de type iterateur sera défini de la manière suivante:
\begin{lstlisting}[language=c++]
class iterateur {
   iterateur& operator++() {   // Version PREFIXEE
      fais_une_iteration();    // PERFORMANTE
      return *this;
   }
   iterateur operator++(int) {           // Version POSTFIXEE
      iterateur vieil_iterateur = *this; // PEU PERFORMANTE
      fais_une_iteration();
      return vieil_iterateur;
   }
}
\end{lstlisting}
\begin{block}{ATTENTION}
\begin{itemize}
   \item{Le paramètre \texttt{int} n'est \textbf{pas utilisé}, il sert uniquement au mécanisme de surcharge}
   \item{Il sera toujours préférable d'utiliser la version \textbf{préfixée} des itérateurs.}
\end{itemize}
\end{block}
\end{frame}

\begin{frame}[fragile=singleslide,shrink=20]
\frametitle {L'opérateur=}

Le même signe = peut servir à l'\textbf{affectation} ou à l'\textbf{initialisation}, or ce sont deux opérations différentes:
\begin{lstlisting}[language=c++]
complexe A = 5;   // Initialisation, appel du constructeur
complexe B = A;   // Initialisation, appel du constructeur de copie
complexe C; 
C = A;            // Affectation, appel de operator=
\end{lstlisting}

L'initialisation correspond à:
\begin{itemize}
\item{Allocation mémoire (ou autres ressources)}
\item{Affectation des membres de l'objet}
\item{Elle est réalisée par \textbf{un constructeur}}
\end{itemize}

L'affectation correspond à:
\begin{itemize}
\item{Vérifications éventuelles de compatibilité}
\item{Affectation des membres de l'objet}
\item{Elle est réalisée par \textbf{un operator=}}
\end{itemize}

\end{frame}

\begin{frame}[fragile=singleslide,shrink=20]
\frametitle {Le quintette infernal 1/3}
\begin{lstlisting}[language=c++]
class tableau {
public:
  tableau(const tableau& t): taille(t.taille) {
    buffer = malloc(t.taille * sizeof(char));
    copie(t.buffer,taille);
  };
  tableau& operator=(const tableau& t): taille(t.taille) {
    if (this !=&t) { // AUTO REFERENCE
       free(buffer);
       buffer = malloc(taille * sizeof(char));
       copie(t.buffer,taille);
       return *this;
    }
  }
  tableau(const tableau&& t) noexcept: 
    taille(t.taille), buffer(t.buffer) {
    t.taille = 0;
    t.buffer = 0;
  };
  tableau& operator=(tableau&& t) noexcept {
    if (this !=&t) { // AUTO REFERENCE
       free(buffer);
       taille = t.taille;
       buffer = t.buffer;
       t.taille = 0;
       t.buffer = nullptr;
       return *this;
    }
  }
  ~tableau() {free buffer;};
\end{lstlisting}
\end{frame}

\begin{frame}[fragile=singleslide,shrink=20]
\frametitle {Le quintette infernal 2/3}
\begin{lstlisting}[language=c++]
private:
  int taille;
  char* buffer;
  void copie(const char* src, int taille) ...
};

void main() {
  tableau tab1(1000);   // Constructeur
  tableau tab2 = tab1;  // Constructeur de copie
  tableau tab3(500);
  tab3 = tab2;          // Operateur=
  tableau tab4 = tab1 + tab2;   // Constructeur de deplacement
  tab3 = tab4 + tab2;           // Operateur= de deplacement
};
\end{lstlisting}

\begin{itemize}
\item{\texttt{Constructeur de copie}  \texttt{malloc} + \texttt{copie} }
\item{\texttt{Opérateur=} \texttt{free} + \texttt{malloc} }
\item{\texttt{Constructeur de déplacement} copie de pointeurs }
\item{\texttt{Opérateur de déplacement} \texttt{free} + copie de pointeurs }
\item{\texttt{Destructeur} \texttt{free} }
\end{itemize}
\end{frame}

\begin{frame}[fragile=singleslide,shrink=20]
\frametitle {Le quintette infernal 3/3}

\textbf{trio en C++, quintette en C++11, mais toujours infernal}

Il est constitué par:
\begin{itemize}
\item{Le constructeur de copie}
\item{L'opérateur d'affectation}
\item{Le constructeur de déplacement}
\item{L'opérateur de déplacement}
\item{Le destructeur}
\end{itemize}

\begin{block}{La règle d'or du trio/quintette infernal}
\begin{itemize}
\item{\textit{La bonne nouvelle}: Le système fournira \textbf{une version par défaut pour ces trois fonctions} }
\item{\textit{la mauvaise nouvelle}: La version par défaut du système \textbf{ne convient pas toujours}, en particulier s'il y a allocation de ressources dans le constructeur}
\item{\textit{La mauvaise nouvelle}: Si je dois écrire l'une des trois/cinq, \textbf{je dois écrire les trois/cinq} ! En effet, si vous fournissez l'un, les duex/quatre autres ne seront pas générés !}
\item{\textit{La bonne nouvelle}: dans du code applicatif, \textbf{les versions par défaut sont la plupart du temps suffisantes}, surtout si on utilise des objets gestionnaires de ressources}
\end{itemize}
\end{block}
\end{frame}

\section{Conversions}
\begin{frame}[fragile=singleslide,shrink=20]
\frametitle {Conversions depuis un type de base vers une classe}

Les constructeurs sont utilisés pour fournir ces conversions:
\begin{lstlisting}[language=c++]
complexe AC = {0, 1};

float CF = 3.0;
complexe CC = CF; // Conversion implicite
AC += CF;

float BF = 2.0;
AC += BF;  // Conversion implicite
\end{lstlisting}
\end{frame}

\begin{frame}[fragile=singleslide,shrink=20]
\frametitle {Conversions depuis une classe vers un type de base}

Les opérateurs de conversion sont utilisés pour cela:
\begin{lstlisting}[language=c++]
class complexe {
     public:
       ...
       operator float() {return r;};
     private:
       ...
     };
     
main() {
  complexe A = (3.5,4.4);
  float Z = (float) A; // Z contient 3.5
\end{lstlisting}
\end{frame}

\begin{frame}
(à suivre)
\end{frame}

\end{document}
