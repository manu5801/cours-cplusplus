\documentclass{beamer}
\usepackage[utf8]{inputenc}
\usepackage{listings}
\usetheme{Warsaw}
\usecolortheme{wolverine}
\usepackage{color}

\definecolor{mygreen}{rgb}{0,0.6,0}
\definecolor{gris}{rgb}{0.9,0.9,0.9}
\definecolor{mymauve}{rgb}{0.58,0,0.82}

\lstset{ %
  backgroundcolor=\color{gris},   % choose the background color; you must add \usepackage{color} or \usepackage{xcolor}
  basicstyle=\footnotesize,        % the size of the fonts that are used for the code
  breakatwhitespace=false,         % sets if automatic breaks should only happen at whitespace
  breaklines=true,                 % sets automatic line breaking
  captionpos=b,                    % sets the caption-position to bottom
  commentstyle=\color{mygreen},    % comment style
  deletekeywords={...},            % if you want to delete keywords from the given language
  escapeinside={\%*}{*)},          % if you want to add LaTeX within your code
  extendedchars=true,              % lets you use non-ASCII characters; for 8-bits encodings only, does not work with UTF-8
  frame=single,	                   % adds a frame around the code
  keepspaces=true,                 % keeps spaces in text, useful for keeping indentation of code (possibly needs columns=flexible)
  keywordstyle=\color{blue},       % keyword style
  language=c++,                 % the language of the code
  otherkeywords={*,...},           % if you want to add more keywords to the set
  %numbers=left,                    % where to put the line-numbers; possible values are (none, left, right)
  %numbersep=5pt,                   % how far the line-numbers are from the code
  %numberstyle=\tiny\color{mygray}, % the style that is used for the line-numbers
  rulecolor=\color{black},         % if not set, the frame-color may be changed on line-breaks within not-black text (e.g. comments (green here))
  showspaces=false,                % show spaces everywhere adding particular underscores; it overrides 'showstringspaces'
  showstringspaces=false,          % underline spaces within strings only
  showtabs=false,                  % show tabs within strings adding particular underscores
  stepnumber=2,                    % the step between two line-numbers. If it's 1, each line will be numbered
  stringstyle=\color{mymauve},     % string literal style
  tabsize=2,	                   % sets default tabsize to 2 spaces
  title=\lstname                   % show the filename of files included with \lstinputlisting; also try caption instead of title
}


\title{Les modèles}
\subtitle{Introduction au C++ et à la programmation objet}
\author{E. Courcelle}\institute{CALMIP, UMS 3669}
\date{Décembre 2019}

\addtobeamertemplate{footline}{\insertframenumber/\inserttotalframenumber}

\begin{document}

\begin{frame}
\titlepage
\end{frame}

\section*{Table des matières}
\begin{frame}
\tableofcontents
\end{frame}

\pgfdeclareimage[height=0.5cm]{logo}{images/cnrs+inpt}
\logo{\pgfuseimage{logo}}

\section{Les modèles}

\subsection{Introduction}

\begin{frame}[fragile=singleslide,shrink=20]
\frametitle {Les modèles pour quoi faire ?}

Ce que nous savons faire:
\begin{itemize}
\item{Exprimer les relations d'agrégation}
\item{Exprimer la relation "Est une sorte de"}
\end{itemize}

Ce dont nous avons encore besoin:
\begin{itemize}
\item{Savoir exprimer la relation "Est une liste de"}
\begin{itemize}
\item{Cela nous permettra d'implémenter des "conteneurs d'objets" avec une structure définie indépendante du type d'objets que l'on met dedans}
\end{itemize}
\end{itemize}

\textbf{Les modèles vont nous le permettre}
\end{frame}

\subsection{Déclarations}

\begin{frame}[fragile=singleslide,shrink=20]
\frametitle {Un modèle de fonction}

La fonction générique ci-dessous renvoie le plus "petit" de ses deux arguments:
\begin{lstlisting}[language=c++]
template <typename T> min(const T& a,const T& b)
{
   if (a > b) return b;
   else return a;
}
\end{lstlisting}

Le code suivant pourra être compilé:
\begin{lstlisting}[language=c++]
int i=4,j=6;
int m = min(i,j);

float x=3.14, y=5.65;
float z= min(x,y);

string s1="houlala", s2="aieaieaie";
string s = min(s1, s2);
\end{lstlisting}

Mais le code suivant \textbf{ne compilera pas}:
\begin{lstlisting}[language=c++]
complexe c1=0, c2=1;
complexe c = min(c1,c2);
\end{lstlisting}

Trois fonctions différentes ont été générées, mais ça coince sur la quatrième.
\end{frame}

\begin{frame}[fragile=singleslide,shrink=20]
\frametitle {Un modèle de classe}

Une nouvelle définition des complexes, qui peut reposer sur différents types de nombres:
\begin{lstlisting}[language=c++]
template<typename NUM=float> class complexe {
public:
 complexe(NUM x=0, NUM y=0);
 complexe(const complexe<NUM> & );
 ~complexe();
 operator NUM();
 complexe<NUM> & operator=(const complexe<NUM> &);
 complexe<NUM> & operator+= (const complexe<NUM> &);
 NUM get_r() const { return r;};
 NUM get_i() const { return i;};
 void set_r(NUM x) { r=x; m_flg=false;};
 void set_i(NUM x) { i=x; m_flg=false;};
 NUM get_m() const;
 static void set_debug() { debflg=true;};
 static void clr_debug() { debflg=false;};

private:
 NUM r;
 NUM i;
 mutable bool m_flg;
 mutable NUM m;
 static bool debflg;
};
\end{lstlisting}
\end{frame}

\begin{frame}[fragile=singleslide,shrink=20]
\frametitle {Un modèle de classe (2)}

Définition d'une fonction-membre en-dehors de la classe:
\begin{lstlisting}[language=c++]
 template <typename NUM> void complexe<NUM>::_calc_module() const {
    m=sqrt(r*r+i*i);
 };
\end{lstlisting}
Il s'agit d'une fonction modèle, membre de la classe \texttt{complexe<NUM>}
\end{frame}

\subsection{Instantiations}

\begin{frame}[fragile=singleslide,shrink=20]
\frametitle {Instantiation du modèle}

On peut alors utiliser notre classe de la manière suivante:
\begin{lstlisting}[language=c++]
main() {
 complexe<> F;          // Utilisation de la valeur par defaut: float
 complexe<double> D;
}
\end{lstlisting}

Ou, sans doute plus lisible grâce à l'instruction C typedef:
\begin{lstlisting}[language=c++]
typedef complexe<> complexe_float;
typedef complexe<double> complexe_double;

main() {
 complexe_float F;
 complexe_double D;
}
\end{lstlisting}
\end{frame}

\begin{frame}[fragile=singleslide,shrink=20]
\frametitle {Paramètres de modèles}

Deux types de paramètres sont utilisables dans les modèles:

\begin{itemize}
\item{\texttt{typename} suivi d'un \textbf{nom de type} (ou de classe) }
\item{\texttt{int} (ou un autre type entier) suivi d'un nombre, qui indique par exemple une dimension}
\end{itemize}

La classe tableau réécrite:
\begin{lstlisting}[language=c++]
template<size_t TAILLE> class tableau {
    private:
       int buffer[TAILLE];
    
    public:
        tableau(){};
};
    
template<size_t TAILLE> tableau<TAILLE>::tableau() {
   for (size_t i=0; i<TAILLE; ++i) { 
      buffer[i] = 0; 
   };
};

main() {
   tableau<100> T1,T2;
   T1 = T2;
}
   
\end{lstlisting}

\begin{itemize}
\item{Pas besoin d'allocation mémoire ni de copie (pas de trio infernal, le compilateur fait ça très bien)}
\item{On a besoin d'un constructeur sinon rien n'est initialisé}
\item{\textbf{ATTENTION ! La taille du tableau est fixée lors de la compilation}}
\end{itemize}
\end{frame}

\begin{frame}[fragile=singleslide,shrink=20]
\frametitle {Paramètres de modèles (2)}

La classe Tableau avec deux paramètres de modèles:
\begin{lstlisting}[language=c++]
template<size_t TAILLE, typename T> class tableau {
    private:
       T buffer[TAILLE];
    
    public:
        tableau(){};
};
    
template<size_t TAILLE, typename T> tableau<TAILLE,T>::tableau() {
   for (size_t i=0; i<TAILLE; ++i) { 
      buffer[i] = 0; 
   };
};

main() {
   tableau<100,float> T1,T2;
   T1 = T2;
}
   
\end{lstlisting}
\end{frame}

\subsection{Spécialisation}

\begin{frame}[fragile=singleslide,shrink=20]
\frametitle {Spécialisation}

Dans ce qui suit, on  suppose que les objets mis dans le tableau sont \textbf{pourvus d'un opérateur inférieur}.

Ecrivons la fonctions maxVal, qui va rechercher la valeur la plus grande du tableau:

\begin{lstlisting}[language=c++]
T template<size_t TAILLE, typename T> tableau<TAILLE>::maxVal() {
   T maxVal = "La valeur la plus grande possible pour T";
   for (size_t i=0; i<TAILLE; ++i) {
      if (buffer[i] < maxVal) maxVal=buffer[i];
   };
   return maxVal;
};
\end{lstlisting}

Comment exprimer "La valeur la plus grande possible pour T" ?

Si T est un type numérique, on peut utiliser les defines trouvés dans \texttt{limits.h}:

\begin{itemize}
\item{ \texttt{SHRT\_MIN SHRT\_MAX} pour des \texttt{short}}
\item{ \texttt{INT\_MIN, INT\_MAX} pour des \texttt{int}}
\item{...}
\end{itemize}
... sauf que ça ne va pas bien s'insérer dans le modèle !

On passe donc par une classe de la bibliothèque standard:
\begin{lstlisting}[language=c++]
T template<size_t TAILLE, typename T> tableau<TAILLE>::maxVal() {
   T maxVal = std::numeric_limits<T>::max();
   for (size_t i=0; i<TAILLE; ++i) {
      if (buffer[i] < maxVal) maxVal=buffer[i];
   };
   return maxVal;
};
\end{lstlisting}
\end{frame}

\begin{frame}[fragile=singleslide,shrink=20]
\frametitle {Spécialisation (2)}

Et pour une classe écrite par moi (\texttt{maClasse}), je \textbf{spécialise} le modèle:

\begin{lstlisting}[language=c++]
template<> struct std::numeric_limits<maClasse> {
   static maClasse min() { return maClasse(...); };
   static maClasse max() { return maClasse(...); };
};
\end{lstlisting}

Voir le fichier d'en-têtes \texttt{limits}

\end{frame}

\begin{frame}[fragile=singleslide,shrink=20]
\frametitle {Spécialisation partielle}

On peut spécialiser partiellement les modèles:

\begin{lstlisting}[language=c++]
template<typename T1, typename T2> class Machin {
...
}

// Les deux parametres ont le meme type
template<typename T> class Machin<T,T> {
...
}

// Le second type est double
template<typename T1> class Machin<T1,double> {
...
}

// On travaille avec des pointeurs
template<typename T1, typename T2> class Machin<T1 *, T2 *> {
...
}
\end{lstlisting}

\end{frame}

\begin{frame}
\frametitle {Tout est fixé à la compilation}

\begin{block}{ATTENTION !}
\begin{itemize}
\item{Tout est fixé (dimensions, types, ...) lors de la compilation}
\item{Le modèle permet d'éviter le copié-collé grâce à la programmation générique}
\item{Les codes des modèles sont écrits \textbf{dans les .hpp}, car le compilateur a besoin du code lors de la compilation}
\end{itemize}
\end{block}

\end{frame}

\begin{frame}
\frametitle {Avantages et inconvénients des modèles}

\begin{block}{LE PLUS}
\begin{itemize}
\item{Tout est fixé (dimensions, types, ...) lors de la compilation:\begin{itemize}\item{le compilateur peut faire tous les tests de cohérence}\end{itemize} }
\item{Le code généré peut être très rapide (pas d'allocation mémoire, ...)}
\item{Une bibliothèque reposant sur des templates sera très simple à installer (juste copier des fichiers, qui sont tous des .hpp)}
\end{itemize}
\end{block}

\begin{block}{LE MOINS}
\begin{itemize}
\item{Tout est fixé (dimensions, types, ...) lors de la compilation:\begin{itemize}\item{le code offrira moins de souplesse (dimensions)}\end{itemize} }
\item{La compilation est plus lente}
\item{L'exécutable généré peut être plus gros}
\end{itemize}
\end{block}

\end{frame}

\begin{frame}
\frametitle {Template libraries}
De nombreuses bibliothèques reposent sur les templates:
\begin{itemize}
\item{La stl: Standard Template Library}
\item{Beaucoup de bibliothèques intégrées à boost}
\item{La bibliothèque de calcul matriciel eigen http://eigen.tuxfamily.org est une bibliothèque de templates}
\end{itemize}
Ces bibliothèques utilisent des techniques de "Template metaprogramming"

cf. https://en.wikipedia.org/wiki/Template\_metaprogramming

\end{frame}

\begin{frame}
(à suivre)
\end{frame}

\end{document}
