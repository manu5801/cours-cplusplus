\documentclass{beamer}
  \usepackage[utf8]{inputenc}
  \usepackage{listings}
  \usetheme{Warsaw}
  \usecolortheme{wolverine}
  
\usepackage{color}

\definecolor{mygreen}{rgb}{0,0.6,0}
\definecolor{gris}{rgb}{0.9,0.9,0.9}
\definecolor{mymauve}{rgb}{0.58,0,0.82}

\lstset{ %
  backgroundcolor=\color{gris},   % choose the background color; you must add \usepackage{color} or \usepackage{xcolor}
  basicstyle=\footnotesize,        % the size of the fonts that are used for the code
  breakatwhitespace=false,         % sets if automatic breaks should only happen at whitespace
  breaklines=true,                 % sets automatic line breaking
  captionpos=b,                    % sets the caption-position to bottom
  commentstyle=\color{mygreen},    % comment style
  deletekeywords={...},            % if you want to delete keywords from the given language
  escapeinside={\%*}{*)},          % if you want to add LaTeX within your code
  extendedchars=true,              % lets you use non-ASCII characters; for 8-bits encodings only, does not work with UTF-8
  frame=single,	                   % adds a frame around the code
  keepspaces=true,                 % keeps spaces in text, useful for keeping indentation of code (possibly needs columns=flexible)
  keywordstyle=\color{blue},       % keyword style
  language=c++,                 % the language of the code
  otherkeywords={*,...},           % if you want to add more keywords to the set
  %numbers=left,                    % where to put the line-numbers; possible values are (none, left, right)
  %numbersep=5pt,                   % how far the line-numbers are from the code
  %numberstyle=\tiny\color{mygray}, % the style that is used for the line-numbers
  rulecolor=\color{black},         % if not set, the frame-color may be changed on line-breaks within not-black text (e.g. comments (green here))
  showspaces=false,                % show spaces everywhere adding particular underscores; it overrides 'showstringspaces'
  showstringspaces=false,          % underline spaces within strings only
  showtabs=false,                  % show tabs within strings adding particular underscores
  stepnumber=2,                    % the step between two line-numbers. If it's 1, each line will be numbered
  stringstyle=\color{mymauve},     % string literal style
  tabsize=2,	                   % sets default tabsize to 2 spaces
  title=\lstname                   % show the filename of files included with \lstinputlisting; also try caption instead of title
}

  \title{Introduction au C++ et à la programmation objets}
  \author{E. Courcelle}\institute{CALMIP, UMS 3669}
  \date{Mai 2016}
  \begin{document}

  \begin{frame}
  \titlepage
  \end{frame}

  \section*{Table des matières}
  \begin{frame}
    \tableofcontents
  \end{frame}

  \section{La programmation orientée objets}

  \subsection{Variables locales, variables globales}
  \begin{frame}
  \frametitle {Une variable globale...}

  Une \textbf{variable globale} est accessible à partir de \\ \textbf{toutes les fonctions du programme}
  
  \end{frame}

  \begin{frame}[fragile=singleslide,shrink=20]
  \frametitle {...Une variable locale}
  
  Une \textbf{variable locale} n'est accessible qu'à partir d'un \textbf{bloc déterminé} \\
  (par exemple une \texttt{fonction}):

  \begin{lstlisting}[language=c++]
  int fonction() {
    %*\textbf{int x=1;}*)
    return 2 * x;         
  }
  \end{lstlisting}
  \end{frame}

  \subsection{L'approche modulaire}
  \begin{frame}
  \frametitle {Interface publique, implémentation cachée}

  \begin{columns}[t]

  \begin{column}{6cm}
  \textbf{Interface du module}
  \begin{itemize}
    \item{Noms des fonctions}
    \item{Ce qu'elles font}
    \item{Ce qui rentre \\ (les paramètres des fonctions)}
    \item{Ce qu'elles renvoient \\ (le type de la valeur de retour)}
  \end{itemize}
  \end{column}

  \begin{column}{6cm}
  \textbf{Implémentation du module}
  \begin{itemize}
     \item{Variables du module \\ ("globales")}
     \item{Code des fonctions}
  \end{itemize}
  \end{column}

  \end{columns}

  \end{frame}
  
  \begin{frame}
  \frametitle {Prototypage et travail en équipe}
  \textbf{L'encapsulation} des données permet:
    \begin{itemize}
    \item De travailler \alert{par prototypage} en commençant \\ par l'interface publique
    \item De travailler en équipe:
    \begin{itemize}
    \item On se met d'accord \alert{sur l'interface}
    \item Chacun écrit l'implémentation \alert{"comme il/elle veut"}
    \end{itemize}
    \end{itemize}
  \end{frame}

  \end{document}
  
