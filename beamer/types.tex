\documentclass{beamer}
\usepackage[utf8]{inputenc}
\usepackage{listings}
\usetheme{Warsaw}
\usecolortheme{wolverine}
\usepackage{color}

\input{lstset.tex}

\title{Les types de base}
\subtitle{Introduction au C++ et à la programmation objet}
\author{E. Courcelle, P. ELYAKIME}\institute{CALMIP UMS 3669, IMFT UMR 5502}
\date{Juin 2022}

\addtobeamertemplate{footline}{\insertframenumber/\inserttotalframenumber}

\begin{document}

\begin{frame}
\titlepage
\end{frame}

\section*{Table des matières}
\begin{frame}
\tableofcontents
\end{frame}

\pgfdeclareimage[height=0.5cm]{logo}{images/cnrs+inpt}
\logo{\pgfuseimage{logo}}

\section{Expressions et instructions}

\begin{frame}[fragile=singleslide,shrink=20]
\frametitle {Les expressions}
\begin{itemize}
\item{Expressions et instructions}
\item{Les instructions conditionnelles}
\item{Les boucles}
\item{\textbf{break} et \textbf{continue}}
\end{itemize}

\end{frame}

\subsection{Expressions et instructions}

\begin{frame}[fragile=singleslide,shrink=20]
\frametitle {Les expressions}
\begin{itemize}
\item{Une expression pas trop compliquée :}
\end{itemize}

\begin{lstlisting}[language=c++]
7;
\end{lstlisting}

\begin{itemize}
\item{Une expression renvoie toujours une valeur : }
\end{itemize}


\begin{lstlisting}[language=c++]
4 + 3; // Renvoie le 7

b = 4 + 3; // met 7 dans la variable b et renvoie aussi la valeur 7

a = (b = 4 + 3); // met 7 dans b et dans a:
\end{lstlisting}

\end{frame}


\begin{frame}[fragile=singleslide,shrink=20]
\frametitle {Instructions simples et complexes}
Une instruction peut être simple, dans ce cas \textbf{elle se termine par un ;}
\begin{lstlisting}[language=c++]
a = 4 + 3;
\end{lstlisting}
Ou alors c'est un \textbf{bloc} et est délimité par $\{$ ou $\}$
\begin{lstlisting}[language=c++]
{
   a = 4 + 3;
   b = 4 - 3;
}
\end{lstlisting}
\begin{block}{Déclarations de variables en C}
Elles se font \textbf{toujours} en début de bloc !
En C++, on déclare les variables n'importe où.
\end{block}
\end{frame}

\subsection{Les instructions conditionnelles}

\begin{frame}[fragile=singleslide,shrink=20]
\frametitle {L'instruction conditionnelle if}

\begin{itemize}
\item{Permet de tester si une condition est vraie ou non}
\end{itemize}


La forme générale:
\begin{lstlisting}[language=c++]
if (expression) 
   instruction1;
else
   instruction2;
\end{lstlisting}

\begin{block}{Commentaires}
\begin{itemize}
\item{\textbf{en C}: Si expression vaut !=0 instruction1, si elle vaut 0 instruction2}
\item{\textbf{en C++}: Si expression vaut \textbf{true} instruction1, si elle vaut \textbf{false} instruction2}
\item{instruction1 et instruction2 peut être un bloc}
\item{else est optionnel}
\end{itemize}
\end{block}
\end{frame}

\begin{frame}[fragile=singleslide,shrink=20]
\frametitle {Plusieurs instructions conditionnelles}

Si on veut tester plusieurs conditions, on peut \textbf{imbriquer les if} mais ce n'est pas très lisible:
\begin{lstlisting}[language=c++]
if (expression1) 
   instruction1;
else if (expression2)
   instruction2;
else if (expression3)
   instruction3;
else
   instruction4;
\end{lstlisting}
\end{frame}

\begin{frame}[fragile=singleslide,shrink=20]
\frametitle {Le switch}

Dans beaucoup de cas, on préfèrera utiliser l'instruction switch plutôt que des if imbriqués:
\begin{lstlisting}[language=c++]
switch(expression) {
    case 1: instruction1;
    case 2: instruction2;
    case 5: instruction3;
    default: instruction4;
}
\end{lstlisting}

\begin{block}{ATTENTION, PIEGE !}
Si expression vaut 1 on exécutera instruction1, instruction2, instruction3, puis instruction4 ! \\
Si vous ne voulez pas cela, vous pouvez utiliser break:
\begin{lstlisting}[language=c++]
switch(expression) {
    case 1: instruction1; break;
    case 2: instruction2; break;
    case 5: instruction3; break;
    default: instruction4;
}
\end{lstlisting}
\end{block}

\begin{block}{Ne pas en abuser !}
L'héritage du C++ permettra de limiter de manière drastique l'utilisation des switches !
\end{block}
\end{frame}

\begin{frame}[fragile=singleslide,shrink=20]
\frametitle {L'instruction ?:;}
C'est un if...else compact, qui peut être utilisé au milieu d'une instruction. \\
Très pratique, à condition de ne pas en abuser car le code risque d'être illisible !

\begin{lstlisting}[language=c++]
// Ce code...
int a;
if ( b == 1 )
{
    a = 3000;
}
else
{
    a = -10;
}

// ...est equivalent a celui-ci:
int A = b==1 ? 3000 : -10;

// Etonnant, non ?
\end{lstlisting}
\end{frame}

\subsection{Les boucles}

\begin{frame}[fragile=singleslide,shrink=20]
\frametitle {Les boucles}

Permet de répéter l'exécution d'une série d'instructions à l'intérieur d'un programme. 
Il existe plusieurs types de boucles en C++ :
\begin{itemize}
\item{for}
\item{while}
\item{do ... while}
\end{itemize}

\end{frame}

\begin{frame}[fragile=singleslide,shrink=20]
\frametitle {for}
\begin{itemize}
\item{Permet de boucler en spécifiant la condition initiale, la condition finale, et incrémentation d'un compteur}
\item{Généralisée en C++ pour balayer un conteneur (par exemple un tableau) avec des itérateur}
\item{La boucle peut ne jamais être exécutée}
\end{itemize}
\begin{lstlisting}[language=c++]
float x = 1;
for (int i=0; i<n; i++)
    x = 2 * x;
\end{lstlisting}
\begin{block}{ATTENTION, PIEGE !}
La forme ci-dessus fonctionne en C++, pas en C. \\
En C on écrira plutôt:
\begin{lstlisting}[language=c++]
int i;
float x = 1;
for (i=0; i<n; i++)
    x = 2 * x;
\end{lstlisting}
\end{block}
\end{frame}

\begin{frame}[fragile=singleslide,shrink=20]
\frametitle {for (en C++11) }
Syntaxe qui provient de la \textbf{STL} : permet d'itérer à travers un \textbf{conteneur}. C'est une instruction de haut niveau
qui utilise la notion d'\textbf{itérateurs}, et qui se révèlera très pratique et très lisible:

\begin{lstlisting}[language=c++]
int somme = 0;
vector<int> mon_vecteur = { 1,2,3,4,5};
for ( auto element : mon_vecteur )
{
    somme += element;
}
\end{lstlisting}
\end{frame}

\begin{frame}[fragile=singleslide,shrink=20]
\frametitle {while}
\begin{itemize}
\item{Permet de boucler jusqu'à ce que la \textbf{condition soit réalisée}}
\item{La condition est évaluée en \textbf{début de boucle}}
\item{La boucle peut ne \textbf{jamais être exécutée}}
\end{itemize}


\begin{lstlisting}[language=c++]
while (c!=' ') {
   c = getchar();
   chaine[i++] = c;
}
\end{lstlisting}
\end{frame}


\begin{frame}[fragile=singleslide,shrink=20]
\frametitle {do... while}
\begin{itemize}
\item{Permet de boucler jusqu'à que la \textbf{condition soit réalisée}}
\item{La condition est évaluée en \textbf{fin de boucle}}
\item{En conséquence, la boucle est exécutée au moins une fois}
\end{itemize}
\begin{lstlisting}[language=c++]
do {
  instruction;
} while (condition);
\end{lstlisting}
\end{frame}

\subsection{break et continue}

\begin{frame}[fragile=singleslide,shrink=20]
\frametitle {Sortir avec break}
break permet de sortir prématurément d'une boucle. \\
Le code C suivant compte le nombre de zéros situés au début d'un tableau:
\begin{lstlisting}[language=c++]
int cpt = 0;
for (cpt = 0; cpt < n; cpt++) {
    if (A[cpt]!=0)
       break;
}
\end{lstlisting}

\begin{block}{ATTENTION}
\begin{itemize}
\item{On ne peut sortir avec break que de la boucle la plus interne.}
\item{break est utilisé seulement dans les switches et dans les boucles}
\end{itemize}
\end{block}
\end{frame}

\begin{frame}[fragile=singleslide,shrink=20]
\frametitle {Sauter des itérations avec continue}
continue permet de sauter des itérations. \\
Imaginons que le code C suivant fait un traitement coûteux pour toutes les cellules non nulles:
\begin{lstlisting}[language=c++]
int cpt = 0;
for (cpt = 0; cpt < n; cpt++) {
    if (A[cpt]==0)
       continue;
    blabla();
    ...faire un traitement complique
}
\end{lstlisting}
\end{frame}

\subsection{Ecrivons du code lisible}

\begin{frame}[fragile=singleslide,shrink=20]
\frametitle {Non au instructions compactées}
Puisqu'une instruction renvoie toujours une valeur, n'importe quelle instruction peut servir de condition dans un if ou une boucle.\\
On peut donc faire plusieurs choses à la fois, par exemple:
\begin{itemize}
\item{Exécuter une instruction en remplissant une variable}
\item{Utiliser la valeur de la variable pour prendre une décision}
\end{itemize}

\textbf{Cela conduit rapidement à du code illisible !}

\begin{lstlisting}[language=c++]
// Plutot que d'ecrire cela:
if (a=sin(theta))
   instruction1;
else
   instruction2;
   
// Ecrivez ce qui suit, un poil plus long mais plus lisible:
a=sin(theta)
if (a!=0)
   instruction1;
else
   instruction2;
\end{lstlisting}
\end{frame}

\section{Le préprocesseur}

\begin{frame}[fragile=singleslide,shrink=20]
\frametitle {Le préprocesseur}

\;


Le préprocesseur est la 1iere étape de la compilation qui procède à des opérations de transformations (rechercher/modifier) dans le code source, comme :

\begin{itemize}
\item{l'inclusion des fichiers d'entêtes }
\item{La définition/lecture des macros}
\item{La compilation conditionnelle}
\end{itemize}


\end{frame}

\subsection{Les includes}

\begin{frame}[fragile=singleslide,shrink=20]
\frametitle {Les fichiers sources en C++}

Un code en C++ est composé des fichiers d'entête \textbf{.h} ou \textbf{.hpp} et des fichiers de codes sources \textbf{.cpp}. \\ \;

Les fichiers d'entête \texttt{.hpp ou .h}, \textbf{utilisés par le compilateur}, contiennent:
\begin{itemize}
\item{Des \textbf{déclarations} de constantes, de fonctions, de classes}
\item{Le code des fonctions déclarées \texttt{inline}}
\item{Des modèles (templates)}
\end{itemize} \;


Les fichiers \texttt{.cpp}, \textbf{utilisés par l'éditeur de lien} contiennent:
\begin{itemize}
\item{Des variables globales}
\item{Les définitions (le code) des fonctions}
\end{itemize}

\end{frame}


\begin{frame}[fragile=singleslide,shrink=20]
\frametitle {Inclusion des fichiers d'entête}

Tous les fichiers d'entête (les includes) se terminant par \texttt{.h} ou \texttt{.hpp} sont inclus au fichier principal pendant l'étape de préprocessing. 

\begin{lstlisting}[language=c++]
#include <iostream>
#include<biblio/chemin/vers/un/fichier.hpp>
#include "mes_fonctions.hpp"
#include "ma_biblio/mes_fonctions.hpp"
\end{lstlisting}

\begin{block}{Où sont les include ?}
Si les includes sont ceux du système (ou de bibliothèques ajoutées), on écrit entre \texttt{<} et \texttt{>}
et le chemin est calculé:

\begin{itemize}
\item{Soit à partir des répertoires systèmes}
\item{Soit à partir des répertoires donnés par le switch de compilateur \texttt{-I}}
\end{itemize}

Si les includes sont ceux de mon application, on écrit entre `` ``
et le chemin est calculé à partir des répertoire courant.
\end{block}

\end{frame}




\begin{frame}[fragile=singleslide,shrink=20]
\frametitle {Les \#include en C++}
Par convention, les include système en C++ n'ont pas d'extension:

\begin{lstlisting}[language=c++]
#include <iostream>
#include <vector>
\end{lstlisting}

Les includes de la bibliothèque standard du C sont utilisables en C++ en :
\begin{itemize}
\item{ajoutant la lettre "c" au début}
\item{retirant l'extension}
\end{itemize}

\begin{lstlisting}[language=c++]
#include <math.h>       // en langage C
#include <cmath>        // la meme chose en langage C++
\end{lstlisting}
\end{frame}

\begin{frame}[fragile=singleslide,shrink=20]
\frametitle {Unicité des headers}
C'est au programmeur de faire en sorte qu'un header ne soit pas inclus plusieurs fois ! (hé oui)
Cela se fait de la manière suivante:

\begin{lstlisting}[language=c++]
#ifndef MON_MODULE
#define MON_MODULE
...
#endif
\end{lstlisting}
\end{frame}

\subsection{Les constantes en langage C}

\begin{frame}[fragile=singleslide,shrink=20]
\frametitle {Les macros}


Le préprocesseur du C effectue des rechercher/remplacer tout au long du code à partir de l'instruction \texttt{\#define} et remplace le symbole par sa valeur. \\
Dans l'exemple ci-dessous, PI sera remplacé par 3.14

\begin{lstlisting}[language=c++]
#define PI 3.14
\end{lstlisting}

Il permet aussi de définir des fonctions, comme la fonction MAX qui sera remplacé par l'instruction qu'il définit dans le code :

\begin{lstlisting}[language=c++]
#define MAX(a,b) a>b?a:b
\end{lstlisting}


\end{frame}

\begin{frame}[fragile=singleslide,shrink=20]
\frametitle {Les constantes en langage C}
Il n'est pas possible de déclarer en langage C des constantes, à part des constantes littérales (cf doc.) pendant la phase de compilation. \\
On utilise donc l'instruction \texttt{\#define} pour construire une constante préprocesseur

\begin{lstlisting}[language=c++]
#define PI 3.14
\end{lstlisting}

\begin{block}{Depuis C++11}
En C++, on peut définir de "vraies" constantes.\\
A partir de la norme C++11, \texttt{\#define} peut être remplacé par \texttt{constexpr}. \\
\texttt{constexpr} est beaucoup plus riche, et permet notamment d'évaluer une fonction à l'étape de la compilation
\end{block}

\begin{lstlisting}[language=c++]
constexpr int a=56;
constexpr float pi=3.14;
\end{lstlisting}


\end{frame}



\begin{frame}[fragile=singleslide,shrink=20]
\frametitle {Les constantes Chaines de caractères}
Le code suivant définit une chaine de caractères:
\begin{lstlisting}[language=c++]
"Bonjour tout le monde"
\end{lstlisting}

\begin{block}{ATTENTION ! PIEGE !}
Une chaine de caractères est un tableau qui se termine par 0 \\
Une constante caractère s'écrit \texttt{'a'}, elle contient \textbf{un} octet.\\
Une chaine de caractères n'ayant qu'\textbf{un seul} caractère s'écrit \textbf{"a"}, \\
elle contient \textbf{deux octets} (le a et le 0).
\end{block}

\begin{block}{En C++...}
En C++, on peut définir de "vraies" chaines de caractère (string)
\end{block}

\end{frame}

\subsection{Compilation conditionnelle}

\begin{frame}[fragile=singleslide,shrink=20]
\frametitle {Compilation conditionnelle}
La compilation conditionnelle permet de compiler ou pas un passage en fonction 
d'une condition:
\begin{lstlisting}[language=c++]
#ifdef GNULINUX
   ... Version de code pour gnu/linux
#else
   ... Version de code pour les autres
#endif
#ifdef WINDOWS
   ... code ajoute pour Windows
#endif
\end{lstlisting}
\end{frame}



\section{Déclarations de variables}



\subsection{Les variables simples}

\begin{frame}[fragile=singleslide,shrink=20]
\frametitle {La portée d'une variable}
La portée d'une variable est la portion de code depuis laquelle \\
on peut utiliser cette variable. \\
\begin{itemize}
\item{Elle commence lorsque la variable est déclarée et se termine à la fin du bloc}
\item{Les blocs internes ont accès à la variable}
\end{itemize}

\begin{lstlisting}[language=c++]
...
{
   ...            // Ici, on ne peut pas utiliser A
   int A = 5;     // Debut de la portee de A
   ...
   {
      ...         // Ici, on peut utiliser A
   }
   ...
}                 // A partir d'ici, on ne peut plus utiliser A 
...
\end{lstlisting}
\end{frame}

\begin{frame}[fragile=singleslide,shrink=20]
\frametitle {Variable globale}
Une variable globale est définit à l'extérieur de toute fonction, de toute classe, ... et est accessible dans tout le programme :
\begin{lstlisting}[language=c++]
..
int A;  			// variable globale

int main() {
	int B;			// variable locale a main
	int C=f1();
};

int f1() {          
	int C=A;		// pas de pb, A est accessible
	C += B;			// ERREUR B n'est pas connu ici
	return C;
}

\end{lstlisting}

\end{frame}


\begin{frame}[fragile=singleslide,shrink=20]
\frametitle {Différences entre C et C++}

\begin{itemize}

\item{En C, les variables \textbf{doivent} être déclarées en début de bloc}
\item{En C++, on peut les déclarer n'importe où}
\end{itemize}  
 
\textbf{Par exemple :} en C++, on peut déclarer des indices de boucle for dans la boucle elle-même: 

\begin{lstlisting}[language=c++]
for (int i=0; i<10; i++) {
   ...
}
\end{lstlisting}
\end{frame}

\begin{frame}[fragile=singleslide,shrink=20]
\frametitle {La structure d'une déclaration de variables}

La forme générale :
\begin{block}{}
[descripteur] [type de base] [déclarateur] [initialisateur]
\end{block}

\begin{itemize}
\item{un descripteur optionnel : const, extern, virtual, ...}
\item{un type de base obligatoire : int, float, double, ...}
\item{un déclarateur obligatoire : nom choisi par le programmeur, un ou plusieurs}
\item{un initialiseur optionnel}
\end{itemize}

\begin{lstlisting}[language=c++]
char* ctbl[] = {"bleu","blanc","rouge"}; 
const int A = 2;
int B; 
\end{lstlisting}
\end{frame}

\subsection{Les types de base}

\begin{frame}[fragile=singleslide,shrink=20]
\frametitle {Les types entiers}
\begin{lstlisting}[language=c++]
short A = 1;
int B = 10;
long B = 100;
long long C = 1000;

unsigned int = A;

\end{lstlisting}
\begin{block}{Remarques}
Le type \textbf{short} prend moins d'octets que \textbf{int}, qui en prend moins que \textbf{long}, qui lui-même en prend moins que \textbf{long long}... \\
Dépends du processeurs 32 ou 64 bits: par exemple pour \textbf{int} 4 ou 8 octets.\\
Il existe des entiers signés (\textbf{signed}) et non signé (\textbf{unsigned}) qui ne sont \textbf{jamais} négatifs.
\end{block}
\end{frame}

\begin{frame}[fragile=singleslide,shrink=20]
\frametitle {D'autres types entiers}
\begin{lstlisting}[language=c++]
char c = 'a';

enum jour_t {lundi=1, mardi, mercredi, jeudi, vendredi, samedi, dimanche};
jour_t jour;
...
if (jour == mercredi) {
   ...
}
\end{lstlisting}
\begin{block}{Les booléens en C}
Il n'y a pas de type booléen en C. \\
Faux s'écrit "entier nul", et vrai s'écrit "entier non nul".
\end{block}
\begin{block}{Les booléens en C++}
En C++, il existe un type booléen. \\
Une variable booléenne peut prendre les valeurs true ou false, \\
ce qui conduit à des codes plus lisibles.
\end{block}
\end{frame}


\begin{frame}[fragile=singleslide,shrink=20]
\frametitle {Le type énumération en C++}

En C, le type \textbf{enum} est un sous-ensemble d'entiers (janvier et lundi valent 1, février et mardi 2, etc), on peut donc comparer des jours avec des mois ! \\
\begin{lstlisting}[language=c++]
enum Jour {lundi=1, mardi, mercredi, jeudi, vendredi, samedi, dimanche};
enum Mois {janvier=1, fevrier, mars, avril, mai, juin, juillet, aout, septembre, octobre, novembre, decembre};
}
\end{lstlisting}

En C++11, le type \textbf{enum class} permet d'exprimer que les jours et les mois sont bien deux objets différents : le code suivant ne compile pas !
\begin{lstlisting}[language=c++]
#include <stdio.h>
enum class Jour {lundi, mardi, mercredi, jeudi, vendredi, samedi, dimanche};
enum class Mois {janvier, fevrier, mars, avril, mai, juin, juillet, aout, septembre, octobre, novembre, decembre};

int main() {
	Jour j = Jour::lundi;
	Mois m = Mois::janvier;
	if (j==m) printf ( "hoho\n"); 
}
\end{lstlisting}



\end{frame}


\begin{frame}[fragile=singleslide,shrink=20]
\frametitle {Les types float}
Il y a quatre types float. Contrairement aux types entiers, leur taille est normalisée, \\ 
ainsi que la manière de coder mantisse et exposant (norme IEE 754)
\begin{itemize}
\item{\textbf{float}. Réel simple précision, codé sur 32 bits}
\item{\textbf{double}. Réel double précision, codé sur 64 bits}
\item{\textbf{long double}. Réel quadruple précision, codé sur 128 bits}
\end{itemize}
\end{frame}

\begin{frame}[fragile=singleslide,shrink=20]
\frametitle {Le type void}
void signifie "vide". \\
On ne peut pas donner ce type à une variable. \\
Mais on peut donner ce type à un pointeur, cela signifie \\ "pointeur vers n'importe quelle variable". \\
On peut aussi donner void comme type de retour d'une fonction, cela signifie \\
"fonction qui ne renvoie rien"
\begin{lstlisting}[language=c++]
void* ptr;
void f(float x);
\end{lstlisting}
\end{frame}

\begin{frame}[fragile=singleslide,shrink=20]
\frametitle {Taille d'une variable d'un type donné}
On a vu à propos des entiers que suivant le système \\ la taille d'un entier peut varier. \\
La fonction \textbf{sizeof} permet de savoir quelle place prend:
\begin{itemize}
\item{une variable}
\item{un type}
\end{itemize}

\begin{lstlisting}[language=c++]
cout << "Taille d'un tres long entier = " << sizeof( long long int );

short s;
cout << "Taille de s = " << sizeof(s);
\end{lstlisting}
\end{frame}

\begin{frame}[fragile=singleslide,shrink=20]
\frametitle {Des types entiers dont on connait la taille}
On peut avoir besoin dans certains cas d'utiliser des entiers dont on connaisse \\
précisément l'occupation mémoire, par exemple pour des questions de portabilité.
\begin{lstlisting}[language=c++]
#include <stdint.h>
int8_t a;
utin8_t au;

int16_t b;
uint16_t bu;

int32_t c;
uint32_t cu;

int64_t d;
uint64_d du;
\end{lstlisting}
\end{frame}

\subsection{Le type auto (c++11)}

\begin{frame}[fragile=singleslide,shrink=20]
\frametitle{Le type auto}

Depuis \textbf{C++11}, il existe le type \textbf{auto} \textbf{utilisé avec un initialiseur} et qui donne à la variable initialisée le même type que la variable à droite du signe $=$.

\begin{lstlisting}[language=c++]
int A = 4;
auto B = A; // B est initialise a partir de A, et a le meme type
\end{lstlisting}
\end{frame}

\subsection{Les types dérivés}

\begin{frame}
\frametitle {Types dérivés}
\begin{itemize}
\item{Tableaux}
\item{Pointeurs}
\item{Structures}
\item{Unions}
\item{Toute combinaison de tous ces types}
\end{itemize}
\end{frame}

\begin{frame}[fragile=singleslide,shrink=20]
\frametitle{Les tableaux}

Une variable ne permet de stocker qu'un seule valeur. Si on veut en stocker plusieurs, on utilise la structure de données appelée \textbf{tableau} qui peut être : 
\begin{itemize}
\item{\textbf{statique} si la taille est fixée une fois pour toute}
\item{\textbf{dynamique} si la taille varie au cours de l'exécution du programme}
\end{itemize}


\begin{block}{Syntaxe de déclaration d'un tableau statique}
type identificateur[taille];\\
identificateur[i];  // pour accéder à l'élément i
\end{block}
où i est un entier, et vaut 0, ..., taille-1  


\begin{block}{Remarques}
\begin{itemize} 
\item{Un tableau statique est une adresse de base vers une zone de la mémoire pour stocker le tableau (= ensemble de "cases" mémoires) }
\item{Selon le type de ses éléments, une case ne fait pas la même taille}
\end{itemize}
\end{block}

\end{frame}

\begin{frame}[fragile=singleslide,shrink=20]
\frametitle{Les tableaux statiques - Exemple}

\begin{lstlisting}[language=c++]
#include <iostream>
using namespace std;

int main() {

	int i;
	int const tailleTableau=10; 
	int tableau[tailleTableau]; 

	for (int i=0; i<tailleTableau; i++ ) {
      tableau[i]=i*i;
      cout<<"Le tableau ["<<i<<"] contient la valeur "<<tableau[i]<<endl;
  }

   return 0;
}
\end{lstlisting}


\end{frame}


\begin{frame}[fragile=singleslide,shrink=20]
\frametitle{Les tableaux}

\begin{block}{Pointeurs et tableaux}
\textbf{Les notions de tableau et de pointeur sont interchangeables}
\end{block}
\begin{block}{Les tableaux en C++}
En C++, la STL propose le type \texttt{vector} ou \texttt{array}, bien plus souples à utiliser que les tableaux C
\end{block}
\end{frame}

\begin{frame}[fragile=singleslide,shrink=20]
\frametitle{Initialiser un tableau statique}
Le tableau \textbf{doit être initialisé}, le C ne le fera pas automatiquement
\begin{lstlisting}[language=c++]
int a[5]   = {34,35,36,37,12};
float x[5] = {3.14,1.5};   // Les 3 derniers elements sont = 0
int b[]    = {1,2,3,4};    // b est de dimension 4
\end{lstlisting}
\end{frame}

\begin{frame}[fragile=singleslide,shrink=20]
\frametitle{Passer un tableau en argument à une fonction}
Comme le C ne connaît que l'adresse de base, le développeur doit \textbf{passer explicitement la taille du tableau} \\
Deux manières de procéder:
\begin{lstlisting}[language=c++]
// Declaration de la fonction: 1ere maniere
void fonction_1(int tab[],int taille) {
   ...
}

// Declaration de la fonction: 2nde maniere
void fonction_2(int* tab,int taille) {
   ...
}

// Utilisation de ces fonctions: pas de difference !
int main() {
   int t[TAILLE];
   
   fonction_1(t,TAILLE);
   fonction_2(t,TAILLE);
 }
\end{lstlisting}
\end{frame}

\begin{frame}[fragile=singleslide,shrink=20]
\frametitle{Les structures}
Un tableau permet de regrouper des données de même type, si nous souhaitons regrouper des données de types différents, nous utiliserons une structure.
\begin{lstlisting}[language=c++]
// Declaration d'une personne
struct personne {
   char nom[20];
   char prenom[20];
   int no_ss;
}

// Initialisation
personne moi = {"Emmanuel","Courcelle",1};

// Acceder aux champs
cout << "Bonjour je m'appelle " << moi.prenom << "\n";
\end{lstlisting}

\begin{block}{Les objets en C++}
Pour déclarer des objets nous utiliserons des \textbf{classes}, \\ qui dérivent directement des structures.
\end{block}
\end{frame}

\section{Les opérateurs}

\begin{frame}[fragile=singleslide,shrink=20]
\frametitle{Les opérateurs}

\begin{itemize}
\item{Les opérateurs d'affectations}
\item{Les opérateurs arithmétique}
\item{Les opérateurs d'incrémentation ou de décrémentation}
\item{Les opérateurs de comparaison}
\item{Les opérateurs logiques}
\item{Les opérateurs pour nombres binaires}
\end{itemize}


\end{frame}

\subsection{Les opérateurs}
\begin{frame}[fragile=singleslide,shrink=20]
\frametitle{Affectation}
\begin{lstlisting}[language=c++]
=
\end{lstlisting}

Copie une variable sur une autre. \\
En C++: Peut être coûteux si la variable est grosse, c'est à dire \textbf{si la variable est un objet}. \\
En C: Peut être un peu coûteux également dans le cas de structures.

\begin{lstlisting}[language=c++]
A = B
\end{lstlisting}

\begin{block}{ATTENTION, PIEGE !}
Si A et B sont des tableaux, B ne sera pas copié sur A ! \\
Simplement l'adresse de base de A sera remplacée par l'adresse de base de B ! \\
\end{block}
\end{frame}

\begin{frame}[fragile=singleslide,shrink=20]
\frametitle{Arithmétique}
\begin{lstlisting}[language=c++]
+ - * / %
+= -= *= %=
\end{lstlisting}

\begin{itemize}
\item{Binaires: + - * / \%}
\item{Unaires: += -= *= /= \%=}
\item{Incrémentation ++ ou décrémentation --}
\end{itemize}

\begin{lstlisting}[language=c++]
A = B + C;

A = A + 4;
A += 4;     // Meme chose !
\end{lstlisting}
\end{frame}

\begin{frame}[fragile=singleslide,shrink=20]
\frametitle{Incrémentation, décrémentation}
\begin{lstlisting}[language=c++]
++i, i++, --i, i--
\end{lstlisting}

\begin{itemize}
\item{Définis sur le type pointeur ou entier}
\item{Servent à itérer dans une boucle: compteurs, indices de tableaux, pointeurs}
\item{En C++, ils sont aussi définis sur les itérateurs}
\end{itemize}
\begin{lstlisting}[language=c++]
for (int i=0; i<1000; ++i) {
    a[i] *= 2;
}
\end{lstlisting}

\begin{block}{ATTENTION ! PIEGE !}
La valeur retournée par ++ ou - - est différente suivant qu'on itère \textbf{avant} ou \textbf{après} la variable.
\begin{lstlisting}[language=c++]
int i = 0;
int j = i++; // j contient 0, i contient 1

int i = 0;
int k = ++i; // k et i contiennent 1
\end{lstlisting}
\end{block}
\end{frame}

\begin{frame}[fragile=singleslide,shrink=20]
\frametitle{Opérateur de comparaison}
\begin{lstlisting}[language=c++]
== , !=
< , <= , > , >=
\end{lstlisting}

\begin{block}{C ou C++}
\begin{itemize}
\item{Ils renvoient 0 ou 1 \textbf{en C}}
\item{Ils renvoie false ou true \textbf{en C++}}
\item{Ils sont utilisés dans les boucles, dans les structures if, etc.}
\end{itemize}
\end{block}
\end{frame}

\begin{frame}[fragile=singleslide,shrink=20]
\frametitle{Opérateur logiques}
\begin{lstlisting}[language=c++]
! , && , ||
\end{lstlisting}
\begin{itemize}
\item{non}
\item{et}
\item{ou}
\end{itemize}
\end{frame}

\begin{frame}[fragile=singleslide,shrink=20]
\frametitle{Opérateurs pour nombres binaires}
\begin{lstlisting}[language=c++]
& | ^ << >>
\end{lstlisting}
\begin{itemize}
\item{ET bit à bit}
\item{OU bit à bit}
\item{NON bit à bit}
\item{Décalage à gauche, donc multiplication par deux}
\item{Décalage à droite, donc division par deux}
\end{itemize}
\begin{block}{C ou C++}
En \textbf{C++}, les opérateurs de décalage prennent le plus souvent \\ 
une autre signification: sortie ou entrée
\end{block}
\end{frame}

\section{Les fonctions}


\begin{frame}[fragile=singleslide,shrink=20]
\frametitle{Les fonctions}

\begin{itemize}
\item{Déclaration et définition d'une fonction}
\item{Récursivité}
\item{Utiliser les fonctions des bibliothèques systèmes}
\item{Fonction main dans un .cpp}
\item{La compilation}
\end{itemize}
\end{frame}


\subsection{Les fonctions}
\begin{frame}[fragile=singleslide,shrink=20]
\frametitle{Les fonctions - déclaration }

La déclaration de l'interface d'une fonction est optionnelle mais très conseillée. On le fait dans les fichiers d'entête \textbf{.hpp} (en C++) ou \textbf{.h} (en C)


\begin{block}{Syntaxe d'une déclaration d'interface}
[type de retour] NomFonction ( Liste des types des arguments )
\end{block}
Le type de retour peut être \textbf{void} si elle ne renvoie rien.\\ 


\begin{block}{Le fichier \texttt{hyperbolique.hpp}}
\begin{lstlisting}[language=c++]
// Une declaration de fonction qui prend un parametre flottant,
// renvoie un parametre flottant
float sinh(float);
float cosh(float);
...
\end{lstlisting}
\end{block}


\end{frame}



\begin{frame}[fragile=singleslide,shrink=20]
\frametitle{Les fonctions - définition}

La définition d'une fonction s'écrit dans le fichier \textbf{.cpp} (en C++) ou \textbf{.c} (en C) :
\begin{itemize}
\item{La déclaration, doit être \textbf{la même} que dans le fichier .h}
\item{Le corps de la fonction contient le code }
\end{itemize}

\begin{block}{Le fichier \texttt{hyperbolique.cpp}}
\begin{lstlisting}[language=c++]
// en debut de fichier, on inclue le header
#include "hyperbolique.hpp"

// La definition de sinh
float sinh(float x) {
   result = exp(x) - exp(-x);
   result /= 2;
   return result;
}
// La definition de cosh
float cosh(float x) {
   result = exp(x) + exp(-x);
   result /= 2;
   return result;
}

\end{lstlisting}
\end{block}
\end{frame}


\begin{frame}[fragile=singleslide,shrink=20]
\frametitle{Les fonctions sont récursives}
Elles peuvent donc s'appeler elles-mêmes:
\begin{lstlisting}[language=c++]
// Version recursive de la fonction factorielle
int factorielle(unsigned int x) {
    if (n==0) 
       return 1;
    else
       return n * factorielle(n - 1);
}
\end{lstlisting}
\end{frame}

\subsection{Compilation séparée}

\begin{frame}[fragile=singleslide,shrink=20]
\frametitle{Les bibliothèques système}
Les include permettent d'utiliser les fonctions définies par le système \\
ou par des éditeurs tiers:

\begin{block}{Les bibliothèques système en C}
\begin{lstlisting}[language=c++]
#include <stdio.h>
\end{lstlisting}
\end{block}

\begin{block}{Les bibliothèques système en C++}
\begin{lstlisting}[language=c++]
#include <iostream>
\end{lstlisting}
\end{block}

\end{frame}



\begin{frame}[fragile=singleslide,shrink=20]
\frametitle{La fonction main dans un .cpp}
Tout exécutable doit avoir une fonction main ! \\
Les fonctions appelées:
\begin{itemize}
\item{doivent être déclarées en incluant les fichiers d'en-têtes}
\item{doivent être résolues à l'édition de liens}
\end{itemize}

\begin{block}{Le fichier cosinh2.cpp}
\begin{lstlisting}[language=c++]
#include <iostream>
#include "hyperbolique.hpp"

int main() {
   float s = sinh(2.0);
   float t = cosh(2.0);
   std.cout << "sinh=" << s << "  cosh=" << t << '\n';
}
\end{lstlisting}
\end{block}
\end{frame}

\begin{frame}[fragile=singleslide,shrink=20]
\frametitle{Pour compiler...}

\begin{block}{Compiler et linker tout d'un coup}
\begin{lstlisting}[language=c++]
g++ -o cosinh2 hyperbolique.cpp cosinh2.cpp
\end{lstlisting}
\end{block}

\begin{block}{Compiler en plusieurs fois}
\begin{lstlisting}[language=c++]
g++ -c -o hyperbolique.o hyperbolique.cpp
g++ -c -o cosinh2.o cosinh2.cpp
g++ -o cosinh2 cosinh2.o hyperbolique.o
\end{lstlisting}
\end{block}

\begin{block}{Faire une bibliothèque et utiliser la bibliothèque}
\begin{lstlisting}[language=c++]
g++ -c -o hyperbolique.o hyperbolique.cpp
ar -c libhyper.a hyperbolique.o
g++ -o cosinh2 cosinh2.cpp -l hyper
\end{lstlisting}
\end{block}

\end{frame}

\section{Les pointeurs}


\begin{frame}[fragile=singleslide,shrink=20]
\frametitle{Les pointeurs}

\begin{itemize}
\item{Les références: un alias}
\item{Les pointeurs}
\item{Le pointeur nul}
\item{Passage de paramètres valeur ou par référence}
\item{Renvoyer une valeur ou une référence}

\end{itemize}
\end{frame}

\subsection{Pointeurs et références}

\begin{frame}[fragile=singleslide,shrink=20]
\frametitle{Les références: un alias}

\begin{lstlisting}[language=c++]
int A=3;
int& a=A;
A = 2 * a;
\end{lstlisting}

\textbf{a} et \textbf{A} ont toujours la même valeur car il s'agit de la même variable. \\
\textbf{Elles sont identiques}

\begin{block}{ATTENTION, Piège !}
\begin{lstlisting}[language=c++]
int &a;    // Ne compile pas ! 
\end{lstlisting}
\end{block}
\end{frame}

\begin{frame}[fragile=singleslide,shrink=20]
\frametitle{Les pointeurs: des variables qui pointent sur d'autres}

\begin{lstlisting}[language=c++]
int A=3;
int* a;
a = &A;
A += 1;
\end{lstlisting}

\textbf{a} contient l'adresse de \textbf{A} \\
*a est la valeur se trouvant à cette adresse, soit A \\
\textbf{*a et A sont égales}

\begin{block}{ATTENTION, Piège !}
\begin{lstlisting}[language=c++]
int *a;    // ca compile, mais risque de plantage !
\end{lstlisting}
\end{block}
\end{frame}

\begin{frame}[fragile=singleslide,shrink=20]
\frametitle{Le pointeur nul: fier de ne pointer sur rien !}

Un pointeur qui ne pointe sur rien, \textbf{c'est très dangereux}.
Le pointeur nul ne pionte sur rien, \textbf{mais on peut le savoir}.

\begin{block}{ATTENTION, Piège !}
Un pointeur doit \textbf{toujours} pointer sur un bloc de variables valide, ou alors être \textbf{nul}.
\end{block}


Comment ça s'écrit ?
\begin{itemize}
\item{en C et en C++ avant C++11: \texttt{NULL}}
\item{en C++11: \texttt{nullptr}}
\end{itemize}

\begin{lstlisting}[language=c++]
int* x=nullptr;
...
if (x != nullptr)
{
   ...traiter le cas pointeur nul
} else {
   ...traiter le cas pointeur valide
}
\end{lstlisting}

\end{frame}

\begin{frame}[fragile=singleslide,shrink=20]
\frametitle{Attention, terrain miné !}

\begin{block}{Ne pas confondre...}
\begin{lstlisting}[language=c++]
int* x=NULL;       // Declaration d'un pointeur vers un entier
...
y = *x + 2;        // Operateur d'indirection !

\end{lstlisting}
\end{block}

\begin{block}{Ne pas confondre non plus...}
\begin{lstlisting}[language=c++]
int& x=A;       // Declaration d'une reference vers un entier
...
int* y = &x;    // Operateur reference !

\end{lstlisting}
\end{block}
\end{frame}

\begin{frame}[fragile=singleslide,shrink=20]
\frametitle{Passage de paramètres par valeur}
Par défaut, les paramètres sont passés \textbf{par valeur} \\
Cela signifie que:
\begin{itemize}
\item{ils sont copiés à l'entrée de la fonction}
\item{\textbf{Cela peut être très lourd !!!!}}
\item{Il ne peut pas y avoir d'effets de bord}
\item{Ce sont des paramètres d'entrée seulement}
\end{itemize}

\begin{block}{Un exemple pas trop dur}
\begin{lstlisting}[language=c++]
void f(int x) {
   x = 0;
}
x = 5;
f(x);   // il y a toujours 5 dans x 
\end{lstlisting}
\end{block}
\end{frame}

\begin{frame}[fragile=singleslide,shrink=20]
\frametitle{Passage de paramètres par référence}
On peut faire en sorte de passer les paramètres par \textbf{référence} \\
Cela signifie que:
\begin{itemize}
\item{Seule leur adresse est copiée à l'entrée de la fonction}
\item{\textbf{Or une adresse c'est tout petit !!!!}}
\item{Il peut y avoir des effets de bord}
\item{Ils peuvent être des paramètres d'entrée ou de sortie}
\end{itemize}

\begin{block}{Passer un paramètre par pointeur en C}
bouh que c'est laid ! \\
A chaque fois qu'on appelle f, il faut se rappeler que le paramètre \\
est passé par référence.
\begin{lstlisting}[language=c++]
void f(int* x) {
   *x = 0;
}
x = 5;
f(&x);   // maintenant il y a 0 dans x
\end{lstlisting}
\end{block}
\end{frame}

\begin{frame}[fragile=singleslide,shrink=20]
\frametitle{Passage de paramètres par référence}
\begin{block}{Passer un paramètre par référence en C++}
Bien plus joli ! \\
L'information "passée par référence" est codée dans le prototype de la fonction \\
\textbf{et c'est tout !}
\begin{lstlisting}[language=c++]
void f(int& x) {
   x = 0;
}
x = 5;
f(x);   // maintenant il y a 0 dans x
\end{lstlisting}
\end{block}
\end{frame}

\begin{frame}[fragile=singleslide,shrink=20]
\frametitle{Passage de paramètres par référence constante}
Avoir le beurre et l'argent du beurre, c'est possible \textbf{en C++} !

\begin{itemize}
\item{Passer une variable par référence, \textbf{c'est bien pour la performance}}
\item{Spécifier que cette variable est constante, \textbf{ça évite les effets de bord}}
\item{Du coup, la variable redevient une \textbf{variable d'entrée} uniquement}
\end{itemize}

\begin{block}{Supprimer les effets de bord}
\begin{lstlisting}[language=c++]
void f(const int& x) {
   x = 0; // NOOOOOOOOOON x est contant, ne compile pas !
}
\end{lstlisting}
\end{block}
\end{frame}

\begin{frame}[fragile=singleslide,shrink=20]
\frametitle{Je passe comment alors ? Pointeur, référence, valeur ?}

\textbf{Pour des variables en entrée}
\begin{itemize}
\item{\textbf{Si la variable est petite}, le passage par valeur est suffisant}
\item{\textbf{Si c'est un gros objet}, préférer le passage par \texttt{const \&}}
\end{itemize}

\textbf{Pour des variables en entrée et en sortie}
\begin{itemize}
\item{\textbf{A chaque fois qu'on peut}, utiliser des \textbf{références}}
\item{\textbf{Quand on n'a pas le choix (\textit{c'est-à-dire jamais})}, utiliser des \textbf{pointeurs}}
\end{itemize}
\end{frame}


\begin{frame}[fragile=singleslide,shrink=20]
\frametitle{Renvoyer une valeur}
Une fonction peut renvoyer une valeur.
\begin{itemize}
\item{Cela peut être très lourd si:
\begin{itemize}
\item{vous utilisez un compilateur tout pourri}
\item{vos objets n'ont pas de constructeur de déplacement (en C++, voir plus loin)}
\item{Car cela signifie une copie de l'objet pour passer "à l'extérieur" de la fonction}
\end{itemize}}
\item{C'est une \texttt{rvalue} (un truc qu'on ne peut pas mettre à \textbf{gauche} de =, du coup on le met à droite.) }
\item Renvoyer une variable ou un objet par valeur est le moyen normal de travailler, \textbf{avec des compilateurs modernes}. Les
compilateurs actuels implémentent en effet le RVO (Return Value Optimization).
\end{itemize}

\end{frame}

\begin{frame}[fragile=singleslide,shrink=20]
\frametitle{Renvoyer une référence}
Une fonction \textbf{en C++} peut renvoyer une référence.
\begin{itemize}
\item{Cela signifie qu'on renvoie juste l'adresse d'un objet}
\item{\textbf{Or une adresse c'est tout petit !!!!}}
\item{\textbf{Attention} à ne pas renvoyer par référence un objet créé dans la fonction !}
\begin{itemize}
\item{ça ne compilera pas, et si ça compile ça plantera}
\end{itemize}
\item{\textbf{Attention} à ne pas renvoyer par référence un paramètre passé en entrée}
\begin{itemize}
\item{ça ne sert à rien}
\item{Si le paramètre est un littéral... boum !}
\end{itemize}
\item Une référence est une \texttt{lvalue} (un truc qu'on peut mettre à \textbf{gauche} de = )
\end{itemize}

\begin{block}{Renvoyer une lvalue en C++}
\textbf{En C++} (mais pas en C) on peut écrire cela, aussi bizarre que ça paraisse:
\begin{lstlisting}[language=c++]
f(x) = y; // ok (en c++)
\end{lstlisting}
\end{block}
\end{frame}

\begin{frame}[fragile=singleslide,shrink=20]
\frametitle{Renvoyer un pointeur}
Une fonction peut renvoyer un pointeur. Un motif de programmation courant:
\begin{itemize}
\item{Allouer dynamiquement de la mémoire dans le corps de la fonction}
\item{Renvoyer le pointeur contenant l'adresse de base}
\end{itemize}

C'est la seule manière de faire en C. \\
Mais si vous faites du C++ (et surtout du C++11):

\begin{itemize}
\item{c'est mal}
\item{c'est moche}
\item{c'est ringard}
\end{itemize}

En C++ moderne: 

\begin{itemize}
\item{On ne fait pas d'allocation dynamique, ou alors on la cache (voir plus loin)}
\item{On renvoie des objets par valeur, \textbf{pas} des pointeurs}
\end{itemize}
\end{frame}

\subsection{Autres choses sur les pointeurs}

\begin{frame}[fragile=singleslide,shrink=20]
\frametitle{Pointeur sur une structure}
\begin{block}{Pointer sur les champs d'une structure (1)}
\begin{lstlisting}[language=c++]
struct personne {
   string nom;
   string prenom;
   int age;
};
personne *p;
(*p).nom    = "Dupont";    // Les () sont indispensables !
(*p).prenom = "Claude";
(*p).age    = 20;
\end{lstlisting}
\end{block}

\begin{block}{Pointer sur les champs d'une structure (2)}
\begin{lstlisting}[language=c++]
struct personne {
   string nom;
   string prenom;
   int age;
};
personne *p;
p->nom    = "Dupont";    // c'est bien plus joli !
p->prenom = "Claude";
p->age    = 20;
\end{lstlisting}
\end{block}
\end{frame}

\section{Les tableaux et pointeurs}

\begin{frame}[fragile=singleslide,shrink=20]
\frametitle{Les tableaux et pointeurs}


\begin{itemize}
\item{Pointeurs et tableaux}
\item{Intialiser, copier des tableaux}
\item{Allocation et désallocation dynamique de la mémoire en C}
\item{Allocation et désallocation dynamique de la mémoire en C++}
\item{les tableaux de la STL}
\end{itemize}



\end{frame}


\begin{frame}[fragile=singleslide,shrink=20]
\frametitle{Pointeurs et tableaux}
Les pointeurs permettent de faire du calcul d'adresse:
\begin{lstlisting}[language=c++]
#define TAILLE 100
int tab[TAILLE];
int* p = tab;     // p pointe vers tab[0]
int *q = tab + 1; // p pointe vers tab[1]
q -= 2;           // HOULALA !!!! q POINTE AVANT LE TABLEAU !!!
q = p + TAILLE - 1; // q pointe vers le dernier element
q = p + TAILLE;     // HOULALA !!!! q POINTE APRES LE TABLEAU !!!
q = p + 5;          // q pointe vers tab[5]
q++;                // q pointe vers tab[6]
cout << q - p;      // 6
\end{lstlisting}
\end{frame}

\begin{frame}[fragile=singleslide,shrink=20]
\frametitle{Initialiser un tableau à 0}
\begin{block}{Première méthode}
\begin{lstlisting}[language=c++]
for (int i=0; i<TAILLE; i++) tab[i] = 0;
\end{lstlisting}
\end{block}
\begin{block}{Seconde méthode}
\begin{lstlisting}[language=c++]
int* q = tab + TAILLE;
for (int* r=tab; r<q; r++) *r = 0;
\end{lstlisting}
\end{block}
\end{frame}

\begin{frame}[fragile=singleslide,shrink=20]
\frametitle{Copier deux tableaux}
\begin{block}{Première méthode}
\begin{lstlisting}[language=c++]
for (int i=0; i<TAILLE; i++) dst[i] = src[i];
\end{lstlisting}
\end{block}
\begin{block}{Seconde méthode}
\begin{lstlisting}[language=c++]
int* s = src;
int* d = dst;
for (int i=0; r<TAILLE; i++) *d++ = *s++;
\end{lstlisting}
\end{block}

\end{frame}

\begin{frame}[fragile=singleslide,shrink=20]
\frametitle{Les void *}
Ce sont des pointeurs vers un type de données non précisé.
\begin{itemize}
\item{Utiles pour échanger des adresses brutes}
\item{Ne permettent pas de faire du calcul d'adresse !}
\end{itemize}
\end{frame}

\subsection{Allocation dynamique de mémoire en C}

\begin{frame}[fragile=singleslide,shrink=20]
\frametitle{Allocation dynamique de la mémoire}
L'entête $<stdlib.h>$ fournit 4 fonctions pour allouer/libérer la mémoire:
\begin{itemize}
\item{\texttt{malloc} pour allouer de la mémoire}
\item{\texttt{calloc} pour allouer et initialiser de la mémoire}
\item{\texttt{realloc} pour réallouer (plus de) mémoire}
\item{\texttt{free} pour rendre la mémoire}
\end{itemize}

\begin{block}{ATTENTION !}
\begin{itemize}
\item{\texttt{malloc}, \texttt{calloc} et \texttt{realloc} renvoient un \texttt{void *}}
\item{Il faudra le convertir pour l'utiliser !}
\item{Si on utilise malloc, Il faudra initialiser le tableau !}
\item{Si l'allocation de mémoire n'a pas fonctionné, renvoie NULL: \textbf{il faut faire le test !}}
\end{itemize}
\end{block}
\end{frame}

\begin{frame}[fragile=singleslide,shrink=20]
\frametitle{malloc}
Allocation de mémoire, sans initialisation.

\begin{lstlisting}[language=c++]
size_t dimension = 1000;
int* tab = (int*) malloc(dimension * sizeof(int));
if ( tab==NULL ) {...erreur... };
cout << tab[0] << '\n';     // renvoie n'importe quoi
\end{lstlisting}
\end{frame}

\begin{frame}[fragile=singleslide,shrink=20]
\frametitle{calloc}
Allocation de mémoire, suivi d'initialisation

\begin{block}{ATTENTION, PIEGE !}
Ne s'utilise pas tout-à-fait comme malloc !
\begin{lstlisting}[language=c++]
size_t dimension = 1000;
int* tab = (int*) calloc(dimension,sizeof(int));
if ( tab==NULL ) {...erreur... };
cout << tab[0] << '\n';    // renvoie 0
\end{lstlisting}
\end{block}
\end{frame}

\begin{frame}[fragile=singleslide,shrink=20]
\frametitle{malloc ou calloc ?}

\begin{itemize}
\item{Dans la vie ordinaire: ni l'un ni l'autre, en C++ on a d'autres solutions}.
\item{Pour un code "HPC" parallèle: préférer malloc car il ne fait pas d'initialisation,
donc il permet de tirer partie de la "first touch policy": il suffit d'initialiser la mémoire
dans les threads pour que la mémoire allouée soit "bien placée" par rapport aux processeurs}
\end{itemize}

\end{frame}

\begin{frame}[fragile=singleslide,shrink=20]
\frametitle{realloc}
Nouvelle allocation de mémoire, plus ou moins grande \\
que l'allocation initiale.

\begin{block}{ATTENTION}
\begin{itemize}
\item{Il est possible que la nouvelle zone soit à un endroit différent !}
\item{Si la seconde zone est plus grande que la première, pas d'initialisation}
\item{Les données sont recopiées de sorte qu'on garde les données qui étaient déjà en mémoire}
\end{itemize}
\begin{lstlisting}[language=c++]
size_t dimension = 1000;
int* tab = (int*) calloc(dimension,sizeof(int));
if ( tab==NULL ) {...erreur... };
...
tab = realloc((void *) tab, 2 * dimension);
if ( tab==NULL ) {...erreur... };
cout << tab[0] << '\n';    // tab[0] pareil avant et apres realloc
\end{lstlisting}
\end{block}
\end{frame}

\subsection{Allocation dynamique de mémoire en C++}

\begin{frame}[fragile=singleslide,shrink=20]
\frametitle{Allocation dynamique de la mémoire en C++}
Un tableau dynamique est un tableau dont le nombre de cases peut varier au cours de l'exécution du programme. Il permet d'ajuster la taille du tableau au besoin du programmeur.\\
Le C++ fournit les opérateurs \textbf{new} et \textbf{delete} pour allouer et désallouer de la mémoire dynamiquement.

\begin{block}{Syntaxe de \textbf{new} et \textbf{delete}}
pointeur = new type[taille];\\
delete[] pointeur;
\end{block}

L'opérateur new[] permet d'allouer une zone mémoire pouvant stocker \textbf{taille} éléments de type \textbf{type}, et retourne l'adresse de cette zone. 

\begin{block}{Attention}
N'oubliez surtout pas les crochets juste après delete, sinon le tableau ne sera pas correctement libéré
\end{block}

\end{frame}

\begin{frame}[fragile=singleslide,shrink=20]
\frametitle{Tableaux à 2 dimensions - Exemple}
\begin{lstlisting}[language=c++]
#include <iostream>
using std::cout;
using std::cin;

int **t;
int nColonnes;
int nLignes;
 
void Free_Tab(); 
 
int main(){
 
  cout << "Nombre de colonnes : ";  cin >> nColonnes;
  cout << "Nombre de lignes : ";  cin >> nLignes;
 
  t = new int* [ nLignes ];
  for (int i=0; i < nLignes; i++)
    t[i] = new int[ nColonnes ];
 

  for (int i=0; i < nLignes; i++)
    for (int j=0; j < nColonnes; j++)
      t[i][j] = i*j;
 
\end{lstlisting}


\end{frame}

\begin{frame}[fragile=singleslide,shrink=20]
\frametitle{Tableaux à 2 dimensions - Exemple}
\begin{lstlisting}[language=c++]

  for (int i=0; i < nLignes; i++) {
 
    for (int j=0; j < nColonnes; j++)
      cout << t[i][j] << " ";
    cout << std::endl;
  }
 
  Free_Tab();
  system("pause>nul");
  return 0;
}
 
void Free_Tab(){
 
  for (int i=0; i < nLignes; i++)
    delete[] t[i];
  delete[] t;
}
\end{lstlisting}

\end{frame}


\subsection{Les tableaux de la STL}

\begin{frame}[fragile=singleslide,shrink=20]
\frametitle{Les tableaux de la STL}

La STL offre des tableaux à 1 dimension (conteneurs) beaucoup plus pratiques à utiliser et qui gèrent eux même la mémoire :  

\begin{itemize}
\item{vector}
\item{array}
\item{...}
\end{itemize}

et pour les tableaux à plusieurs dimensions voir avec la librairie \textbf{BOOST}

\end{frame}



\end{document}
