\documentclass{beamer}
\usepackage[utf8]{inputenc}
\usepackage{listings}
\usetheme{Warsaw}
\usecolortheme{wolverine}
\usepackage{color}

\definecolor{mygreen}{rgb}{0,0.6,0}
\definecolor{gris}{rgb}{0.9,0.9,0.9}
\definecolor{mymauve}{rgb}{0.58,0,0.82}

\lstset{ %
  backgroundcolor=\color{gris},   % choose the background color; you must add \usepackage{color} or \usepackage{xcolor}
  basicstyle=\footnotesize,        % the size of the fonts that are used for the code
  breakatwhitespace=false,         % sets if automatic breaks should only happen at whitespace
  breaklines=true,                 % sets automatic line breaking
  captionpos=b,                    % sets the caption-position to bottom
  commentstyle=\color{mygreen},    % comment style
  deletekeywords={...},            % if you want to delete keywords from the given language
  escapeinside={\%*}{*)},          % if you want to add LaTeX within your code
  extendedchars=true,              % lets you use non-ASCII characters; for 8-bits encodings only, does not work with UTF-8
  frame=single,	                   % adds a frame around the code
  keepspaces=true,                 % keeps spaces in text, useful for keeping indentation of code (possibly needs columns=flexible)
  keywordstyle=\color{blue},       % keyword style
  language=c++,                 % the language of the code
  otherkeywords={*,...},           % if you want to add more keywords to the set
  %numbers=left,                    % where to put the line-numbers; possible values are (none, left, right)
  %numbersep=5pt,                   % how far the line-numbers are from the code
  %numberstyle=\tiny\color{mygray}, % the style that is used for the line-numbers
  rulecolor=\color{black},         % if not set, the frame-color may be changed on line-breaks within not-black text (e.g. comments (green here))
  showspaces=false,                % show spaces everywhere adding particular underscores; it overrides 'showstringspaces'
  showstringspaces=false,          % underline spaces within strings only
  showtabs=false,                  % show tabs within strings adding particular underscores
  stepnumber=2,                    % the step between two line-numbers. If it's 1, each line will be numbered
  stringstyle=\color{mymauve},     % string literal style
  tabsize=2,	                   % sets default tabsize to 2 spaces
  title=\lstname                   % show the filename of files included with \lstinputlisting; also try caption instead of title
}


\title{Les types de base}
\subtitle{Introduction au C++ et à la programmation objet}
\author{E. Courcelle}\institute{CALMIP, UMS 3669}
\date{Mai 2016}

\begin{document}

\begin{frame}
\titlepage
\end{frame}

\section*{Table des matières}
\begin{frame}
\tableofcontents
\end{frame}

\pgfdeclareimage[height=0.5cm]{logo}{images/cnrs+inpt}
\logo{\pgfuseimage{logo}}

\section{Expressions et instructions}

\subsection{Expressions et instructions}

\begin{frame}[fragile=singleslide,shrink=20]
\frametitle {Une expression pas trop compliquée}
\begin{lstlisting}[language=c++]
7;
\end{lstlisting}
\end{frame}

\begin{frame}[fragile=singleslide,shrink=20]
\frametitle {Une expression renvoie toujours une valeur}
L'expression suivante renvoie 7:
\begin{lstlisting}[language=c++]
4 + 3;
\end{lstlisting}

L'expression suivante:
\begin{itemize}
\item{met 7 dans la variable b}
\item{renvoie \textbf{aussi} la valeur 7}
\end{itemize}
\begin{lstlisting}[language=c++]
b = 4 + 3;
\end{lstlisting}

Si bien que l'expression suivante met 7 dans b \textbf{et} dans a:
\begin{lstlisting}[language=c++]
a = (b = 4 + 3);
\end{lstlisting}
\end{frame}

\begin{frame}[fragile=singleslide,shrink=20]
\frametitle {Instructions simples et complexes}
Une instruction peut être simple, dans ce cas \textbf{elle se termine par un ;}
\begin{lstlisting}[language=c++]
a = 4 + 3;
\end{lstlisting}
Ou alors c'est un \textbf{bloc}, il est délimité par { ou }
\begin{lstlisting}[language=c++]
{
   a = 4 + 3;
   b = 4 - 3;
}
\end{lstlisting}
\begin{block}{Déclarations de variables en C}
Elles se font \textbf{toujours} en début de bloc !
En C++, on déclare les variables n'importe où.
\end{block}
\end{frame}

\subsection{Les instructions conditionnelles}

\begin{frame}[fragile=singleslide,shrink=20]
\frametitle {L'instruction conditionnelle if}

La forme générale:
\begin{lstlisting}[language=c++]
if (expression) 
   instruction1;
else
   instruction2;
\end{lstlisting}

\begin{block}{Commentaires}
\begin{itemize}
\item{\textbf{en C}: Si expression vaut 1 instruction1, si elle vaut 2 instruction2}
\item{\textbf{en C++}: Si expression vaut \textbf{true} instruction1, si elle vaut \textbf{false} instruction2}
\item{instruction1 et instruction2 peut être un bloc}
\item{else est optionnel}
\end{itemize}
\end{block}
\end{frame}

\begin{frame}[fragile=singleslide,shrink=20]
\frametitle {Les if imbriqués}

On peut écrire aussi, même si ce n'est pas très lisible:
\begin{lstlisting}[language=c++]
if (expression1) 
   instruction1;
else if (expression2)
   instruction2;
else if (expression3)
   instruction3;
else
   instruction4;
\end{lstlisting}
\end{frame}

\begin{frame}[fragile=singleslide,shrink=20]
\frametitle {Le switch}

Dans beaucoup de cas, on préfèrera utiliser l'instruction switch plutôt que des if imbriqués:
\begin{lstlisting}[language=c++]
switch(expression) {
    case 1: instruction1;
    case 2: instruction2;
    case 5: instruction3;
    default: instruction4;
}
\end{lstlisting}

\begin{block}{ATTENTION, PIEGE !}
Si expression vaut 1 on exécutera instruction1, instruction2, instruction3, puis instruction4 ! \\
Si vous ne voulez pas cela, vous pouvez utiliser break:
\begin{lstlisting}[language=c++]
switch(expression) {
    case 1: instruction1; break;
    case 2: instruction2; break;
    case 5: instruction3; break;
    default: instruction4;
}
\end{lstlisting}
\end{block}

\begin{block}{Ne pas en abuser !}
L'héritage du C++ permettra de limiter de manière drastique l'utilisation des switches !
\end{block}
\end{frame}

\begin{frame}[fragile=singleslide,shrink=20]
\frametitle {L'instruction ?:;}
C'est un if...else compact, qui peut être utilisé au milieu d'une instruction. \\
Très pratique, à condition de ne pas en abuser car le code risque d'être illisible !

\begin{lstlisting}[language=c++]
// Ce code...
int a;
if ( b == 1 )
{
	a = 3000;
}
else
{
	a = -10;
}

// ...est equivalent a celui-ci:
int A = b==1 ? 3000 : -10;

// Etonnant, non ?
\end{lstlisting}
\end{frame}

\subsection{Les boucles}

\begin{frame}[fragile=singleslide,shrink=20]
\frametitle {for}
\begin{itemize}
\item{Permet de boucler en spécifiant condition initiale, condition finale, et incrémentation d'un compteur}
\item{Généralisée en C++ pour balayer un conteneur (par exemple un tableau) avec des itérateur}
\item{La boucle peut ne jamais être exécutée}
\end{itemize}
\begin{lstlisting}[language=c++]
float x = 1;
for (int i=0; i<n; i++)
    x = 2 * x;
\end{lstlisting}
\begin{block}{ATTENTION, PIEGE !}
La forme ci-dessus fonctionne en C++, pas en C. \\
En C on écrira plutôt:
\begin{lstlisting}[language=c++]
int i;
float x = 1;
for (i=0; i<n; i++)
    x = 2 * x;
\end{lstlisting}
\end{block}
\end{frame}

\begin{frame}[fragile=singleslide,shrink=20]
\frametitle {while}
\begin{itemize}
\item{Permet de boucler jusqu'à que la condition soit réalisée}
\item{La boucle peut ne jamais être exécutée}
\end{itemize}
\begin{lstlisting}[language=c++]
while (c!=' ') {
   c = getchar();
   chaine[i++] = c;
}
\end{lstlisting}
\end{frame}

\begin{frame}[fragile=singleslide,shrink=20]
\frametitle {do... while}
\begin{itemize}
\item{Permet de boucler jusqu'à que la condition soit réalisée}
\item{La condition est évaluée en fin de boucle}
\item{En conséquence, la boucle est exécutée au moins une fois}
\end{itemize}
\begin{lstlisting}[language=c++]
do {
  instruction;
} while (condition);
\end{lstlisting}
\end{frame}

\subsection{break et continue}

\begin{frame}[fragile=singleslide,shrink=20]
\frametitle {Sortir avec break}
break permet de sortir prématurément d'une boucle. \\
Le code C suivant compte le nombre de zéros situés au début d'un tableau:
\begin{lstlisting}[language=c++]
int cpt = 0;
for (cpt = 0; cpt < n; cpt++) {
    if (A[cpt]!=0)
       break;
}
\end{lstlisting}

\begin{block}{ATTENTION}
\begin{itemize}
\item{On ne peut sortir avec break que de la boucle la plus interne.}
\item{break est utilisé seulement dans les switches et dans les boucles}
\end{itemize}
\end{block}
\end{frame}

\begin{frame}[fragile=singleslide,shrink=20]
\frametitle {Sauter des itérations avec continue}
continue permet de sauter des itérations. \\
Imaginons que le code C suivant fait un traitement coûteux pour toutes les cellules non nulles:
\begin{lstlisting}[language=c++]
int cpt = 0;
for (cpt = 0; cpt < n; cpt++) {
    if (A[cpt]==0)
       continue;
    blabla();
    ...faire un traitement complique
}
\end{lstlisting}
\end{frame}

\subsection{Ecrivons du code lisible}

\begin{frame}[fragile=singleslide,shrink=20]
\frametitle {Non à l'obfuscation !}
Puisqu'une instruction renvoie toujours une valeur, n'importe quelle instruction \\
peut servir de condition dans un if ou une boucle.\\
On peut donc faire plusieurs choses à la fois:
\begin{itemize}
\item{Exécuter une instruction, par exemple remplir une variable}
\item{Utiliser la valeur de la variable pour prendre une décision}
\end{itemize}

\textbf{Cela conduit rapidement à du code illisible !}

\begin{lstlisting}[language=c++]
// Plutot que d'ecrire cela:
if (a=sin(theta))
   instruction1;
else
   instruction2;
   
// Ecrivez ce qui suit, un poil plus long mais plus lisible:
a=sin(theta)
if (a!=0)
   instruction1;
else
   instruction2;
\end{lstlisting}
\end{frame}

\section{Le préprocesseur}

\subsection{Les constantes en langage C}

\begin{frame}[fragile=singleslide,shrink=20]
\frametitle {Les constantes en langage C}
Le préprocesseur du C est un programme séparé du compilateur. \\
Il effectue des rechercher/remplacer tout au long du code.

il n'y a pas de constantes en C. \\
Mais l'instruction \#define permet d'utiliser des symboles. \\
Dans l'exemple ci-dessous,PI sera remplacé par 3.14

\begin{lstlisting}[language=c++]
#define PI 3.14
\end{lstlisting}

\begin{block}{En C++...}
En C++, on peut définir de "vraies" constantes.
\end{block}

\end{frame}

\begin{frame}[fragile=singleslide,shrink=20]
\frametitle {Les constantes Chaines de caractères}
Le code suivant définit une chaine de caractères:
\begin{lstlisting}[language=c++]
"Bonjour tout le monde"
\end{lstlisting}

\begin{block}{ATTENTION ! PIEGE !}
Une chaine de caractères est un tableau qui se termine par 0 \\
Une constante caractère s'écrit 'a', elle contient un octet.\\
Une chaine de caractères n'ayant qu'un seul caractère s'écrit "a", \\
elle contient deux octets (le a et le 0).
\end{block}

\begin{block}{En C++...}
En C++, on peut définir de "vraies" chaines de caractère (string)
\end{block}

\end{frame}

\subsection{Compilation conditionnelle}

\begin{frame}[fragile=singleslide,shrink=20]
\frametitle {Compilation conditionnelle}
bla bla
\end{frame}

\subsection{Les includes}

\begin{frame}[fragile=singleslide,shrink=20]
\frametitle {\#include}
Voir plus loin
\end{frame}

\section{Déclarations de variables}

\subsection{Les variables simples}

\begin{frame}[fragile=singleslide,shrink=20]
\frametitle {La portée d'une variable}
La portée d'une variable est la portion de code depuis laquelle \\
on peut utiliser cette variable. \\
\begin{itemize}
\item{Elle commence lorsque la variable est déclarée et se termine à la fin du bloc}
\item{Les blocs internes ont accès à la variable}
\end{itemize}

\begin{lstlisting}[language=c++]
...
{
   ...            // Ici, on ne peut pas utiliser A
   int A = 5;     // Debut de la portee de A
   ...
   {
      ...         // Ici, on peut utiliser A
   }
   ...
}                 // A partir d'ici, on ne peut plus utiliser A 
...
\end{lstlisting}
\end{frame}

\begin{frame}[fragile=singleslide,shrink=20]
\frametitle {Différences entre C et C++}

\begin{block}{En C...}
En C, les variables doivent être déclarées en début de bloc. \\
En C++, on peut les déclarer simplement lorsqu'on en a besoin.
\end{block}
\begin{block}{En C++...}
En C++, on peut déclarer des indices de boucle for de dans la boucle elle-même:
\begin{lstlisting}[language=c++]
for (int i=0; i<10; i++) {
   ...
}
\end{lstlisting}
\end{block}
\end{frame}

\subsection{Les types de base}

\begin{frame}[fragile=singleslide,shrink=20]
\frametitle {Les types entiers}
\begin{lstlisting}[language=c++]
short A = 1;
int B = 10;
long B = 100;
long long C = 1000;

unsigned int = A;

\end{lstlisting}
\begin{block}{ATTENTION ! PIEGE !}
Tout ce qu'on sait, c'est que \textbf{short} prend moins d'octets que \textbf{int}, \\
qui en prend moins que \textbf{long}, qui lui-même en prend moins que \textbf{long long}. \\
Mais d'une machine à l'autre, les tailles des entiers \\ 
ne sont pas obligatoirement les mêmes. \\
Les types unsigned ne sont jamais négatifs.
\end{block}
\end{frame}

\begin{frame}[fragile=singleslide,shrink=20]
\frametitle {D'autres types entiers}
\begin{lstlisting}[language=c++]
char c = 'a';

enum jour_t {lundi=1, mardi, mercredi, jeudi, vendredi, samedi, dimanche};
jour_t jour;
...
if (jour == mercredi) {
   ...
}
\end{lstlisting}
\begin{block}{Les booléens en C}
Il n'y a pas de type booléen en C. \\
Faux s'écrit "entier nul", et vrai s'écrit "entier non nul".
\end{block}
\begin{block}{Les booléens en C++}
En C++, il existe un type booléen. \\
Une variable booléenne peut prendre les valeurs true ou false, \\
ce qui conduit à des codes plus lisibles.
\end{block}
\end{frame}

\begin{frame}[fragile=singleslide,shrink=20]
\frametitle {Les types float}
Il y a quatre types float. Contrairement aux types entiers, leur taille est normalisée, \\ 
ainsi que la manière de coder mantisse et exposant (norme IEE 754)
\begin{itemize}
\item{\textbf{float}. Réel simple précision, codé sur 32 bits}
\item{\textbf{double}. Réel double précision, codé sur 64 bits}
\item{\textbf{long double}. Réel quadruple précision, codé sur 128 bits}
\end{itemize}
\end{frame}

\begin{frame}[fragile=singleslide,shrink=20]
\frametitle {Le type void}
void signifie "vide". \\
On ne peut pas donner ce type à une variable. \\
Mais on peut donner ce type à un pointeur, cela signifie \\ "pointeur vers n'importe quelle variable". \\
On peut aussi donner void comme type de retour d'une fonction, cela signifie \\
"fonction qui ne renvoie rien"
\begin{lstlisting}[language=c++]
void* ptr;
void f(float x);
\end{lstlisting}
\end{frame}

\begin{frame}[fragile=singleslide,shrink=20]
\frametitle {Taille d'une variable d'un type donné}
On a vu à propos des entiers que suivant le système \\ la taille d'un entier peut varier. \\
La fonction \textbf{sizeof} permet de savoir quelle place prend:
\begin{itemize}
\item{une variable}
\item{un type}
\end{itemize}

\begin{lstlisting}[language=c++]
cout << "Taille d'un entier quadruple precision = " << sizeof( long long int );

short s;
cout << "Taille de s = " << sizeof(s);
\end{lstlisting}
\end{frame}

\begin{frame}[fragile=singleslide,shrink=20]
\frametitle {Des types entiers dont on connait la taille}
On peut avoir besoin dans certains cas d'utiliser des entiers dont on connaisse \\
précisément l'occupation mémoire, par exemple pour des questions de portabilité.
\begin{lstlisting}[language=c++]
#include <stdint.h>
int8_t a;
utin8_t au;

int16_t b;
uint16_t bu;

int32_t c;
uint32_t cu;

int64_t d;
uint64_d du;
\end{lstlisting}
\end{frame}

\subsection{Le type auto (c++11)}

\begin{frame}[fragile=singleslide,shrink=20]
\frametitle{Le type auto}
\textbf{En C++11 uniquement}, permet de donner à une variable \\ 
le même type qu'une autre variable:
\begin{lstlisting}[language=c++]
int A = 4;
auto B = A; // B est initialise a partir de A, et a le meme type
\end{lstlisting}
\end{frame}

\subsection{Les types dérivés}

\begin{frame}
\frametitle {Types dérivés}
\begin{itemize}
\item{Tableaux}
\item{Pointeurs}
\item{Structures}
\item{Unions}
\item{Toute combinaison de tous ces types}
\end{itemize}
\end{frame}

\begin{frame}[fragile=singleslide,shrink=20]
\frametitle{Les tableaux}
\begin{block}{Les tableaux en C}
En C, un tableau est:
\begin{itemize}
\item{Une adresse de base}
\item{Un type d'éléments, permettant de connaitre la taille de la cellule}
\item{Une zone de la mémoire pour stocker le tableau}
\end{itemize}
\end{block}
\begin{lstlisting}[language=c++]
#define TAILLE 10
int A[TAILLE];
\end{lstlisting}
\begin{block}{ATTENTION, PIEGE}
\textbf{C'est au programmeur de se souvenir de la taille du tableau}
\end{block}
\begin{block}{Pointeurs et tableaux}
\textbf{Les notions de tableau et de pointeur sont interchangeables}
\end{block}
\begin{block}{Les tableaux en C++}
En C++, la stdlib propose les vecteurs, bien plus souples à utiliser que les tableaux C
\end{block}
\end{frame}

\begin{frame}[fragile=singleslide,shrink=20]
\frametitle{Initialiser le tableau}
Le tableau \textbf{doit être initialisé}, le C ne le fera pas automatiquement
\begin{lstlisting}[language=c++]
int a[5]   = {34,35,36,37,12};
float x[5] = {3.14,1.5};   // Les 3 derniers elements sont = 0
int b[]    = {1,2,3,4};    // b est de dimension 4
\end{lstlisting}
\end{frame}

\begin{frame}[fragile=singleslide,shrink=20]
\frametitle{Passer le tableau à une fonction}
Comme le C ne connait que l'adresse de base, le développeur doit \\
\textbf{passer explicitement la taille du tableau} \\
Deux manières de procéder:
\begin{lstlisting}[language=c++]
// Declaration de la fonction: 1ere maniere
void fonction_1(int tab[],int taille) {
   ...
}

// Declaration de la fonction: 2nde maniere
void fonction_2(int* tab,int taille) {
   ...
}

// Utilisation de ces fonctions: pas de difference !
int main() {
   int t[TAILLE];
   
   fonction_1(t,TAILLE);
   fonction_2(t,TAILLE);
 }
\end{lstlisting}
\end{frame}

\begin{frame}[fragile=singleslide,shrink=20]
\frametitle{Les structures}
Permet de mettre dans une entité unique plusieurs données de types différents.
\begin{lstlisting}[language=c++]
// Declaration d'une personne
struct personne {
   char nom[20];
   char prenom[20];
   int no_ss;
}

// Initialisation
personne moi = {"Emanuel","Courcelle",1};

// Acceder aux champs
cout << "Bonjour je m'appelle " << moi.prenom << "\n";
\end{lstlisting}

\begin{block}{Les objets en C++}
Pour déclarer des objets nous utiliserons des \textbf{classes}, \\ qui dérivent directement des structures.
\end{block}
\end{frame}

\section{Les opérateurs}
\subsection{Les opérateurs}
\begin{frame}[fragile=singleslide,shrink=20]
\frametitle{Affectation}
\begin{lstlisting}[language=c++]
=
\end{lstlisting}

Copie une variable sur une autre. \\
Peut être coûteux si la variable est grosse. \\
\begin{lstlisting}[language=c++]
A = B
\end{lstlisting}
\begin{block}{ATTENTION, PIEGE !}
Si A et B sont des tableaux, B ne sera pas copié sur A ! \\
Simplement l'adresse de base de A sera remplacée par l'adresse de base de B ! \\
\end{block}
\end{frame}

\begin{frame}[fragile=singleslide,shrink=20]
\frametitle{Arithmétiques}
\begin{lstlisting}[language=c++]
+ - * / %
+= -= *= %=
\end{lstlisting}

\begin{itemize}
\item{Binaires: + - * / \%}
\item{Unaires: += -= *= /= \%=}
\item{Incrémentation ++ ou décrémentation --}
\end{itemize}

\begin{lstlisting}[language=c++]
A = B + C;

A = A + 4;
A += 4;     // Meme chose !
\end{lstlisting}
\end{frame}

\begin{frame}[fragile=singleslide,shrink=20]
\frametitle{Incrémentation, décrémentation}
\begin{lstlisting}[language=c++]
++i, i++, --i, i--
\end{lstlisting}

\begin{itemize}
\item{Définis sur le type pointeur ou entier}
\item{Servent à itérer dans une boucle: compteurs, indices de tableaux, pointeurs}
\item{En C++, ils sont aussi définis sur les itérateurs}
\end{itemize}
\begin{lstlisting}[language=c++]
for (int i=0; i<1000; ++i) {
    a[i] *= 2;
}
\end{lstlisting}

\begin{block}{ATTENTION ! PIEGE !}
La valeur retournée par ++ ou - - est différente suivant qu'on itère avant ou après la variable.
\begin{lstlisting}[language=c++]
int i = 0;
int j = i++; // j contient 0, i contient 1

int i = 0;
int k = ++i; // k et i contiennent 1
\end{lstlisting}
\end{block}
\end{frame}

\begin{frame}[fragile=singleslide,shrink=20]
\frametitle{Opérateur de comparaison}
\begin{lstlisting}[language=c++]
== , !=
< , <= , > , >=
\end{lstlisting}

\begin{block}{C ou C++}
\begin{itemize}
\item{Ils renvoient 0 ou 1 \textbf{en C}}
\item{Ils renvoie false ou true \textbf{en C++}}
\item{Ils sont utilisés dans les boucles, dans les structures if, etc.}
\end{itemize}
\end{block}
\end{frame}

\begin{frame}[fragile=singleslide,shrink=20]
\frametitle{Opérateur logiques}
\begin{lstlisting}[language=c++]
! , && , ||
\end{lstlisting}
\begin{itemize}
\item{non}
\item{et}
\item{ou}
\end{itemize}
\end{frame}

\begin{frame}[fragile=singleslide,shrink=20]
\frametitle{Opérateurs pour nombres binaires}
\begin{lstlisting}[language=c++]
& | ^ << >>
\end{lstlisting}
\begin{itemize}
\item{ET bit à bit}
\item{OU bit à bit}
\item{NON bit à bit}
\item{Décalage à gauche, donc multiplication par deux}
\item{Décalage à droite, donc division par deux}
\end{itemize}
\begin{block}{C ou C++}
En \textbf{C++}, les opérateurs de décalage prennent le plus souvent \\ 
une autre signification: sortie ou entrée
\end{block}
\end{frame}

\section{Les fonctions}
\subsection{Les fonctions}
\begin{frame}[fragile=singleslide,shrink=20]
\frametitle{Les fonctions}
Deux parties, la première est \textbf{optionnelle}:
\begin{itemize}
\item{La déclaration: l'interface}
\item{La définition: L'implémentation, le code de la fonction}
\end{itemize}

\begin{block}{La déclaration}
Se trouve dans le fichier .hpp \\
Comprend trois parties:
\begin{itemize}
\item{Le type de retour, éventuelement void}
\item{Le nom de la fonction}
\item{Les types de arguments}
\end{itemize}
\end{block}

\begin{block}{La définition}
Se trouve dans le fichier .cpp \\
Comprend deux parties:
\begin{itemize}
\item{La déclaration, doit être \textbf{la même} que dans le fchier .h}
\item{Le corps de la fonction: le code}
\end{itemize}
\end{block}
\end{frame}

\begin{frame}[fragile=singleslide,shrink=20]
\frametitle{Les fonctions sont récursives}
Elles peuvent donc s'appeler elles-mêmes:
\begin{lstlisting}[language=c++]
// Version recursive de la fonction factorielle
int factorielle(int x) {
    if (n==1) 
       return 1;
    else
       return n * factorielle(n - 1);
}
\end{lstlisting}
\end{frame}

\subsection{Compilation séparée}

\begin{frame}[fragile=singleslide,shrink=20]
\frametitle{Les bibliothèques système}
Les include permettent d'utiliser les fonctions définies par le système \\
ou par des éditeurs tiers:

\begin{block}{Les bibliothèques système en C}
\begin{lstlisting}[language=c++]
#include stdio.h>
\end{lstlisting}
\end{block}

\begin{block}{Les bibliothèques système en C++}
\begin{lstlisting}[language=c++]
#include <iostream>
\end{lstlisting}
\end{block}


\end{frame}

\begin{frame}[fragile=singleslide,shrink=20]
\frametitle{Les déclarations dans le .hpp}
Les déclarations de fonctions se trouvent  \\
dans les fichiers \texttt{.hpp } (en C++) \\
ou dans les fichiers \texttt{.h} (en C)

\begin{block}{Le fichier \texttt{hyperbolique.hpp}}
\begin{lstlisting}[language=c++]
// Une declaration de fonction qui prend un parametre flottant,
// renvoie un parametre flottant
float sinh(float);
float cosh(float);
...
\end{lstlisting}
\end{block}
\end{frame}

\begin{frame}[fragile=singleslide,shrink=20]
\frametitle{Les définitions dans le .cpp}
Les définitions des fonctions se trouvent \\
dans les fichiers \texttt{.cpp} (en C++) \\
ou dans les fichiers \texttt{.c} (en C)

\begin{block}{Le fichier \texttt{hyperbolique.cpp}}
\begin{lstlisting}[language=c++]
// en debut de fichier, on inclue le header
#include "hyperbolique.hpp"

// La definition de sinh
float sinh(float x) {
   result = exp(x) - exp(-x);
   result /= 2;
   return result;
}
// La definition de cosh
float cosh(float x) {
   result = exp(x) + exp(-x);
   result /= 2;
   return result;
}

\end{lstlisting}
\end{block}
\end{frame}

\begin{frame}[fragile=singleslide,shrink=20]
\frametitle{La fonction main dans un .cpp}
Tout exécutable doit avoir une fonction main ! \\
Les fonctions appelées:
\begin{itemize}
\item{doivent être déclarées en incluant les fichiers d'e"n-têtes}
\item{n'ont pas besoin d'être définies dans l'unité de compilation considérée}
\item{doivent être résolues à l'édition de liens}
\end{itemize}

\begin{block}{Le fichier cosinh2.cpp}
\begin{lstlisting}[language=c++]
#include <iostream>
#include "hyperbolique.hpp"

int main() {
   float s = sinh(2.0);
   float t = cosh(2.0);
   std.cout << "sinh=" << s << "  cosh=" << t << '\n';
}
\end{lstlisting}
\end{block}
\end{frame}

\begin{frame}[fragile=singleslide,shrink=20]
\frametitle{Pour compiler...}

\begin{block}{Compiler et linker tout d'un coup}
\begin{lstlisting}[language=c++]
c++ -o cosinh2 hyperbolique.cpp cosinh2.cpp
\end{lstlisting}
\end{block}

\begin{block}{Compiler en plusieurs fois}
\begin{lstlisting}[language=c++]
c++ -c -o hyperbolique.o hyperbolique.cpp
c++ -c -o cosinh2.o cosinh2.cpp
c++ -o cosinh2 cosinh2.o hyperbolique.o
\end{lstlisting}
\end{block}

\begin{block}{Faire une bibliothèque et utiliser la bibliothèque}
\begin{lstlisting}[language=c++]
c++ -c -o hyperbolique.o hyperbolique.cpp
ar -c libhyper.a hyperbolique.o
c++ -o cosinh2 cosinh2.cpp -l hyper
\end{lstlisting}
\end{block}

\end{frame}

\subsection{Pointeurs et références}

\begin{frame}[fragile=singleslide,shrink=20]
\frametitle{Les références: un alias}

\begin{lstlisting}[language=c++]
int A=3;
int& a=A;
A = 2 * a;
\end{lstlisting}

\textbf{a} et \textbf{A} ont toujours la même valeur car il s'agit de la même variable. \\
\textbf{Elles sont identiques}

\begin{block}{ATTENTION, Piège !}
\begin{lstlisting}[language=c++]
int &a;    // Ne compile pas ! 
\end{lstlisting}
\end{block}
\end{frame}

\begin{frame}[fragile=singleslide,shrink=20]
\frametitle{Les pointeurs: des variables qui pointent sur d'autres}

\begin{lstlisting}[language=c++]
int A=3;
int* a;
a = &A;
A += 1;
\end{lstlisting}

\textbf{a} contient l'adresse de \textbf{A} \\
*a est la valeur se trouvant à cette adresse, soit A \\
\textbf{*a et A sont égales}

\begin{block}{ATTENTION, Piège !}
\begin{lstlisting}[language=c++]
int *a;    // ca compile, mais risque de plantage !
\end{lstlisting}
\end{block}
\end{frame}

\begin{frame}[fragile=singleslide,shrink=20]
\frametitle{Attention, terrain miné !}

\begin{block}{Ne pas confondre...}
\begin{lstlisting}[language=c++]
int* x=NULL;       // Declaration d'un pointeur vers un entier
...
y = *x + 2;        // Operateur d'indirection !

\end{lstlisting}
\end{block}

\begin{block}{Ne pas confondre non plus...}
\begin{lstlisting}[language=c++]
int& x=A;       // Declaration d'une reference vers un entier
...
int* y = &x;    // Operateur reference !

\end{lstlisting}
\end{block}
\end{frame}

\begin{frame}[fragile=singleslide,shrink=20]
\frametitle{Passage de paramètres par valeur}
Par défaut, les paramètres sont passés \textbf{par valeur} \\
Cela signifie que:
\begin{itemize}
\item{ils sont copiés à l'entrée de la fonction}
\item{\textbf{Cela peut être très lourd !!!!}}
\item{Il ne peut pas y avoir d'effets de bord}
\item{Ce sont des paramètres d'entrée seulement}
\end{itemize}

\begin{block}{Un exemple pas trop dur}
\begin{lstlisting}[language=c++]
void f(int x) {
   x = 0;
}
x = 5;
f(x);   // il y a toujours 5 dans x 
\end{lstlisting}
\end{block}
\end{frame}

\begin{frame}[fragile=singleslide,shrink=20]
\frametitle{Passage de paramètres par référence}
On peut faire en sorte de passer les paramètres par \textbf{référence} \\
Cela signifie que:
\begin{itemize}
\item{Seule leur adresse est copiée à l'entrée de la fonction}
\item{\textbf{Or une adresse c'est tout petit !!!!}}
\item{Il peut y avoir des effets de bord (ou pas...)}
\item{Ils peuvent être des paramètres d'entrée ou de sortie}
\end{itemize}

\begin{block}{Passer un paramètre par adresse en C}
bouh que c'est laid ! \\
A chaque fois qu'on appelle f, il faut se rappeler que le paramètre \\
est passé par référence.
\begin{lstlisting}[language=c++]
void f(int* x) {
   *x = 0;
}
x = 5;
f(&x);   // maintenant il y a 0 dans x
\end{lstlisting}
\end{block}
\end{frame}

\begin{frame}[fragile=singleslide,shrink=20]
\frametitle{Passage de paramètres par référence}
\begin{block}{Passer un paramètre par référence en C++}
Bien plus joli ! \\
L'information "passé par référence" est codée dans le prototype de la fonction \\
\textbf{et c'est tout !}
\begin{lstlisting}[language=c++]
void f(int& x) {
   x = 0;
}
x = 5;
f(x);   // maintenant il y a 0 dans x
\end{lstlisting}
\end{block}
\end{frame}

\begin{frame}[fragile=singleslide,shrink=20]
\frametitle{Renvoyer une valeur}
Une fonction peut renvoyer une valeur.
\begin{itemize}
\item{Cela signifie une copie pour passer "à l'extérieur" de la fonction}
\item{\textbf{Cela peut être très lourd !!!!}}
\item{Cest une \texttt{rvalue} (un truc qu'on met à \textbf{droite} de = ) }
\end{itemize}

\begin{block}{Renvoyer une lvalue en C}
On n'a jamais vu \textbf{en C} une ligne de code aussi bizarre:
\begin{lstlisting}[language=c++]
f(x) = y;  // nawak !
y = f(x);  // ok
\end{lstlisting}
\end{block}
Par contre, une fonction C a le droit de renvoyer un pointeur \\
ça contient une adresse, mais c'est une \texttt{rvalue} !
\end{frame}

\begin{frame}[fragile=singleslide,shrink=20]
\frametitle{Renvoyer une référence}
Une fonction peut renvoyer une référence.
\begin{itemize}
\item{Cela signifie qu'on renvoie juste l'adresse d'un objet}
\item{\textbf{Or une adresse c'est tout petit !!!!}}
\item{Attention à ne pas renvoyer un objet créé dans la fonction !}
\item{Attention à ne pas renvoyer par référence un paramètre passé en entrée}
\begin{itemize}
\item{ça ne sert à rien}
\item{Si le paramètre est un littéral... boum !}
\end{itemize}
\item{\textbf{Cela peut être très lourd !!!!} (quoique, les compilos modernes savent optimiser) }
\item{Cest une \texttt{lvalue} (un truc qu'on met à \textbf{gauche} de = )}
\end{itemize}

\begin{block}{Renvoyer une lvalue en C++}
\textbf{En C++} (mais pas en C) on peut écrire cela, aussi bizarre que ça paraisse:
\begin{lstlisting}[language=c++]
f(x) = y; // ok (en c++)
\end{lstlisting}
\end{block}
\end{frame}

\begin{frame}[fragile=singleslide,shrink=20]
\frametitle{Passage de paramètres par référence constante}
Avoir le beurre et l'argent du beurre, c'est possible \textbf{en C++} !

\begin{itemize}
\item{Passer une variable par référence, \textbf{c'est bien pour la performance}}
\item{Spécifier que cette variable est constante, \textbf{ça évite les effets de bord}}
\item{Du coup, la variable redevient une \textbf{variable d'entrée} uniquement}
\end{itemize}

\begin{block}{Supprimer les effets de bord}
\begin{lstlisting}[language=c++]
void f(const int& x) {
   x = 0; // NOOOOOOOOOON x est contant, ne compile pas !
}
\end{lstlisting}
\end{block}
\end{frame}

\begin{frame}[fragile=singleslide,shrink=20]
\frametitle{Je passe comment alors ? Pointeur, référence, valeur ?}

\textbf{Pour des variables en entrée}
\begin{itemize}
\item{\textbf{Si la variable est petite}, le passage par valeur est suffisant}
\item{\textbf{Si c'est un gros objet}, préférer le passage par \texttt{const \&}}
\end{itemize}

\textbf{Pour des variables en entrée et en sortie}
\begin{itemize}
\item{\textbf{A chaque fois qu'on peut}, utiliser des \textbf{références}}
\item{\textbf{Quand on n'a pas le choix}, utiliser des \textbf{pointeurs}}
\end{itemize}
\end{frame}

\begin{frame}
(à suivre)
\end{frame}

\end{document}
